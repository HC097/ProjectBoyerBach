\documentclass[9pt]{article}
\usepackage{amsmath, amssymb, geometry, graphicx, float, fancyhdr}
\usepackage[most]{tcolorbox}

\geometry{margin=1in}
\pagestyle{fancy}
\fancyhf{}
\rhead{Su2025MATH4991 Lecture Notes}
\lhead{Chapter 12 - The Latin West}
\cfoot{\thepage}

\title{Chapter 12 -- The Latin West}
\author{Mr. Harley Caham Combest}
\date{June 2025}

\begin{document}
\maketitle

\section*{I. Cultural Invocation}

\begin{itemize}
    \item \textbf{Civilization:} Latin Christendom (Frankish, Carolingian, Italian, English, Iberian, Parisian, Papal)
    \item \textbf{Time Period:} c. 500--1450 AD
    \item \textbf{Figures:} 
    Boethius (quadrivium, transmission),  
    Isidore of Seville (encyclopedism),  
    Gerbert of Aurillac (abacus, numerals),  
    Adelard of Bath (translation, Euclid),  
    Fibonacci (Liber Abaci, sequence),  
    Jordanus Nemorarius (letters, mechanics),  
    Nicole Oresme (graphs, series, exponents)
\end{itemize}

\begin{quote}
\itshape
In the Latin West, mathematics did not rise from triumph —  
It rose from ruin. \\
The empire had fallen. The libraries were ash. The language of proof lay buried in forgotten scripts. \\
Yet the monks copied. The friars taught. The merchants calculated.\\
At first, they remembered only fragments — Boethius’ definitions, Isidore’s divisions, the abacus, the fingers, the stars.\\
But memory became method.\\
Through Spain and Sicily, Arabic numerals crossed into Latin ink.  
Through Fibonacci, algebra entered the counting house.\\
Through Jordanus and Campanus, Euclid spoke once more.  
Through Oresme, the curve became concept — the infinite, measurable.\\
The West did not create mathematics anew.  
It \textbf{reassembled} it.  
Piece by piece, parchment by parchment, they restored the language of number.\\
It was not brilliance, but \textit{fidelity}, that preserved the flame.\\
This was not the mathematics of power.  
This was the mathematics of survival.\\
And it made the Renaissance possible.
\end{quote}

\begin{center}
    \includegraphics[scale=0.6]{cover3.png}
\end{center}

\newpage

\section*{II. Faces of the Era}

\vspace{0.5em}

\subsection*{Boethius (c. 480–524 CE)}

\subsubsection*{The Quadrivium and Transmission of Greek Thought}

Boethius preserved fragments of Greek mathematical thought during the early collapse of Roman institutions.  
His treatises on arithmetic and geometry, though basic, defined the quadrivium: arithmetic, geometry, music, and astronomy — a structure that endured for centuries in monastic education.

\vspace{0.5em}

\subsection*{Isidore of Seville (c. 560–636 CE)}

\subsubsection*{Etymologiae and Medieval Compendia}

A compiler of vast encyclopedic knowledge, Isidore’s \textit{Etymologiae} included a section on mathematics.  
Though lacking in rigor, it preserved Roman definitions and structure, ensuring that number and proportion remained in the memory of the West.

\vspace{0.5em}

\subsection*{Gerbert of Aurillac (Pope Sylvester II) (c. 946–1003 CE)}

\subsubsection*{Abacus and Early Use of Hindu-Arabic Numerals}

Educated in Spain and Italy, Gerbert taught arithmetic using a counting board and possibly Hindu-Arabic digits — centuries ahead of their widespread use.  
He exemplifies the blend of ecclesiastical power and mathematical curiosity in the early Holy Roman Empire.

\vspace{0.5em}

\subsection*{Adelard of Bath (c. 1075–1160 CE)}

\subsubsection*{Translation of Euclid and Arabic Knowledge}

Adelard journeyed through the Islamic world and returned with translations of astronomical tables and Euclid’s \textit{Elements}.  
He helped transfer the intellectual legacy of Arabic science to Latin Europe, laying groundwork for future formal education in geometry.

\vspace{0.5em}

\subsection*{Fibonacci (Leonardo of Pisa) (c. 1170–1250 CE)}

\subsubsection*{Liber Abaci and Introduction of Hindu-Arabic Arithmetic}

Fibonacci’s \textit{Liber Abaci} (1202) introduced the Hindu-Arabic numeral system to a Latin commercial audience.  
Its problems in currency, interest, and trade made mathematics immediately relevant — and its rabbit sequence entered mathematical folklore.

\subsubsection*{Liber Quadratorum and Number Theory}

In his later work, Fibonacci explored indeterminate problems and quadratic identities, often inspired by Diophantus and Islamic sources.  
This marked one of the earliest examples of original mathematical development in the Christian West.

\vspace{0.5em}

\subsection*{Jordanus Nemorarius (c. 1225–1260 CE)}

\subsubsection*{Letter-Based Arithmetic and Ratio Theory}

Jordanus used letters to express general arithmetical rules — anticipating symbolic algebra.  
His \textit{Arithmetica} and \textit{De numeris datis} emphasized logical clarity and generality in number manipulation.

\subsubsection*{Statics and the Law of the Inclined Plane}

In \textit{De ratione ponderis}, Jordanus formulated a correct law of the inclined plane using proportional reasoning — a precursor to later physics.

\vspace{0.5em}

\subsection*{Campanus of Novara (fl. c. 1260 CE)}

\subsubsection*{Translation and Consolidation of Euclid}

Campanus produced the most authoritative medieval edition of Euclid’s \textit{Elements}, synthesizing Latin and Arabic sources.  
His version became the standard text into the Renaissance.

\vspace{0.5em}

\subsection*{Nicole Oresme (c. 1323–1382 CE)}

\subsubsection*{Graphical Analysis and the Merton Rule}

Oresme pioneered the graphical representation of change — using a velocity-time diagram to justify the mean velocity theorem.  
This visual approach prefigured ideas central to calculus.

\subsubsection*{Infinite Series and Divergence of the Harmonic Series}

He provided geometric demonstrations for convergent series like \( \sum \frac{n}{2^n} = 2 \), and proved the divergence of the harmonic series using grouped comparisons.

\subsubsection*{Compound Proportions and Rational Powers}

Oresme’s use of labeled squares to model fractional exponents (like \( x^{3/2} \)) shows an advanced grasp of abstract relationships between magnitude and proportion — centuries before exponents became symbolic.

\newpage

\section*{III. Works of the Era}

\subsection*{Boethius — \textit{De Institutione Arithmetica} (c. 500 CE)}

\subsubsection*{Definition-Based Arithmetic in the Roman Tradition}

Based largely on Nicomachus of Gerasa, this text codified number theory into simple definitions and properties, emphasizing even/odd, prime/composite, and perfect numbers.  
It served as a mathematical cornerstone for centuries in Latin Europe despite its lack of proofs.

\subsubsection*{Framing the Quadrivium}

Boethius outlined the four mathematical arts — arithmetic, geometry, music, and astronomy — as integral to liberal education, preserving a Platonic-Pythagorean hierarchy of knowledge.

\vspace{0.5em}

\subsection*{Isidore of Seville — \textit{Etymologiae} (c. 630 CE)}

\subsubsection*{Mathematics as Lexical Memory}

Book III of this encyclopedic text defined basic mathematical concepts for monastic scribes.  
Though derivative and etymological, it ensured continuity of terminology and preserved Roman frames of mathematical classification.

\vspace{0.5em}

\subsection*{Adelard of Bath — \textit{Elements of Euclid} (trans. c. 1120–1142 CE)}

\subsubsection*{Earliest Latin Translation of Euclid from Arabic Sources}

Adelard’s work brought the structure of Euclidean geometry back into Western thought via Arabic intermediaries.  
It reintroduced deductive reasoning in geometry to Latin education, though its impact became widespread only in the following century.

\vspace{0.5em}

\subsection*{Fibonacci — \textit{Liber Abaci} (1202, revised 1228 CE)}

\subsubsection*{Introduction of Hindu-Arabic Numerals to Latin Commerce}

This work explained place-value notation and arithmetic operations using Hindu-Arabic digits.  
It applied these techniques to business, weights, currencies, and partnerships — embedding arithmetic into mercantile reality.

\subsubsection*{Development of Recursive Sequences}

The book contains the famous rabbit-pair problem, which generates the sequence now named after Fibonacci:  
\[
F_n = F_{n-1} + F_{n-2}
\]

\vspace{0.5em}

\subsection*{Fibonacci — \textit{Liber Quadratorum} (c. 1225 CE)}

\subsubsection*{Number Theory and Diophantine Techniques}

This text explores indeterminate equations and identities such as:
\[
(a^2 + b^2)(c^2 + d^2) = (ac + bd)^2 + (ad - bc)^2
\]
Fibonacci investigates square decompositions and rational solutions — building on Greek and Islamic traditions.

\vspace{0.5em}

\subsection*{Jordanus Nemorarius — \textit{Arithmetica} (c. 1230 CE)}

\subsubsection*{Letter-Based Arithmetic and General Theorems}

This work introduced the use of letters to denote general numbers, enabling abstract rules of arithmetic — a vital step toward symbolic algebra.

\vspace{0.5em}

\subsection*{Jordanus Nemorarius — \textit{De ratione ponderis} (c. 1235 CE)}

\subsubsection*{Statics and Early Mechanics}

Jordanus presents a correct rule for the inclined plane: force along the slope is proportional to the vertical height.  
This is one of the earliest quantitative treatments of physical equilibrium in the West.

\vspace{0.5em}

\subsection*{Campanus of Novara — \textit{Elements of Euclid} (compiled c. 1260 CE)}

\subsubsection*{Standard Medieval Euclid Edition}

This authoritative version, synthesizing Adelard’s and Arabic sources, became the dominant Euclid in Western Europe — cited and reprinted well into the Renaissance.

\vspace{0.5em}

\subsection*{Nicole Oresme — \textit{Algorismus proportionum} (c. 1350–1360 CE)}

\subsubsection*{Rules for Compound Proportions and Rational Powers}

This text offers verbal and diagrammatic representations of powers such as \( x^{3/2} \), called “one and one-half proportion.”  
It explores fractional multiplication using geometric partitioning.

\vspace{0.5em}

\subsection*{Nicole Oresme — \textit{De latitudinibus formarum} (c. 1355–1361 CE)}

\subsubsection*{Graphical Representation of Variable Change}

Oresme introduces diagrams for speed, time, and intensity — anticipating Cartesian graphs.  
He shows that uniform acceleration yields a triangular area, giving rise to the Merton Rule.

\subsubsection*{Geometric Proof of Series Convergence and Divergence}

He evaluates series like:
\[
\sum_{n=1}^\infty \frac{n}{2^n} = 2, \quad \sum_{n=1}^\infty \frac{3n}{4^n} = \frac{4}{3}
\]
and proves the divergence of \( \sum \frac{1}{n} \) using grouped lower bounds.



\newpage

\section*{II. Historical Overview}

The mathematical awakening of the Latin West was not born of invention, but of convergence —  
a slow, uneven assimilation of number systems, instruments, and philosophical inheritances.

\subsection*{1. The Spread of Hindu-Arabic Numerals}

The Hindu-Arabic numeral system, with its base-10 structure and positional zero, reached the Latin West by way of Arabic treatises.  
It did not arrive cleanly — it trickled in.

\begin{itemize}
    \item \textbf{Early Hints (10\textsuperscript{th} c.):} A Spanish copy of Isidore’s \textit{Etymologiae} (c. 992) includes nine numerals — but no zero.
    \item \textbf{Gerbert of Aurillac (c. 1000):} Possibly used an abacus with Arabic apices. His exact knowledge remains debated.
    \item \textbf{Fibonacci (1202):} The \textit{Liber Abaci} gave systematic treatment to Hindu-Arabic numerals, advocating their use in trade and everyday reckoning.
\end{itemize}

Despite these appearances, the system faced cultural resistance for centuries. The positional value of a zero was conceptually foreign, and the script unfamiliar.  
It would take the practical victories of the algorists — and eventually the printing press — to seal its adoption.

\subsection*{2. Algorists and Abacists: A Two-Century Rivalry}

Once Hindu-Arabic digits entered Latin arithmetic, they spawned a rivalry:  
\textbf{Algorists} — who used pen-and-paper algorithms — versus \textbf{Abacists} — who relied on counters and traditional methods.

\begin{itemize}
    \item \textbf{Abacists} used counting boards, Roman numerals, and physical manipulation.
    \item \textbf{Algorists} embraced written calculations, place-value notation, and eventually algebraic reasoning.
    \item \textbf{Notable Algorists:} Sacrobosco (\textit{Algorismus vulgaris}), Alexandre of Villedieu (\textit{Carmen de Algorismo}), Fibonacci.
\end{itemize}

The conflict was not merely methodological — it was institutional and pedagogical.  
Merchant schools trained abacists. Universities and clerics trained algorists.

\begin{figure}[H]
\centering
\includegraphics[scale=0.16]{12extra1.png}
\caption{Woodcut from the \textit{Margarita Philosophica} (1503): Arithmetic instructs algorist and abacist}
\end{figure}

For over 200 years, both systems coexisted. Only in the 16\textsuperscript{th} century did the algorist method, aided by the press and symbolic algebra, finally prevail.

\subsection*{3. Aristotle in the Latin West: What Was Kept, What Was Recast}

When Greek mathematics returned to Latin Europe — via Arabic translations — it came wrapped in Aristotle.  
But the Latin Scholastics adopted Aristotle selectively, reshaping his framework to serve Christian metaphysics and emerging science.

\begin{itemize}
    \item \textbf{Kept:}
    \begin{itemize}
        \item The theory of proportion (from \textit{Elements V}, filtered through Aristotle’s logic)
        \item The distinction between potential and actual infinity
        \item The categorization of forms, causes, and motion
    \end{itemize}
    \item \textbf{Modified:}
    \begin{itemize}
        \item Motion was redefined: The Merton School and Oresme introduced quantification and mean speed — which Aristotle never formalized
        \item Infinity was cautiously explored: Oresme and others treated series as summable without metaphysical commitment
    \end{itemize}
    \item \textbf{Discarded or Challenged:}
    \begin{itemize}
        \item Aristotle’s rejection of the vacuum and inertia
        \item His linear theory of motion speed (proportional to force/resistance)
        \item His disdain for applied mathematics as inferior to natural philosophy
    \end{itemize}
\end{itemize}

In the end, Aristotle served as both foundation and foil. The Scholastics debated him, extended him, and quietly surpassed him — especially in kinematics and proto-calculus.



\newpage

\section*{V. Problem–Solution Cycle}

\subsection*{Problem 1: Fibonacci’s Exchange Puzzle (c. 1202 CE)}

\textbf{Statement.} If 1 solidus imperial (12 deniers imperial) is worth 31 deniers Pisan, how many deniers Pisan should one obtain for 11 deniers imperial?

\textbf{Solution.} Apply proportional reasoning:
\[
\frac{12}{31} = \frac{11}{x} \Rightarrow 12x = 341 \Rightarrow x = \frac{341}{12} = 28\frac{5}{12}
\]

\textbf{Answer.} \(28\frac{5}{12}\) deniers Pisan.

% No figure for this one.

\newpage

\subsection*{Problem 2: Fibonacci’s Rabbit Sequence (c. 1202 CE)}

\textbf{Statement.} Beginning with one pair of rabbits, and assuming each pair produces a new pair every month starting from their second month, how many pairs exist after one year?

\textbf{Solution.} The Fibonacci sequence is defined recursively:
\[
F_1 = 1,\quad F_2 = 1,\quad F_n = F_{n-1} + F_{n-2}
\]
Compute through month 12:
\[
F_{12} = 144
\]

\textbf{Answer.} 144 pairs of rabbits after 12 months.

\begin{figure}[H]
\centering
\includegraphics[width=0.55\textwidth]{12extra3.png}
\caption{Fibonacci’s rabbit-pair problem and recursive solution in the \textit{Liber Abaci} (c. 1202)}
\end{figure}

\newpage

\subsection*{Problem 3: Campanus–Jordanus Angle Trisection (c. 1250 CE)}

\textbf{Statement.} Describe the method attributed to Campanus and Jordanus for trisecting angle \( \angle AOB \).

\textbf{Solution.} Construct the following:
\begin{itemize}
  \item Let \( OA = OB \)
  \item Draw a radius \( OC \perp OB \)
  \item Through point \( A \), draw line \( AED \) so that \( DE = OA \)
  \item Draw line \( OF \parallel AED \)
\end{itemize}
Then:
\[
\angle FOB = \frac{1}{3} \angle AOB
\]

\textbf{Answer.} Trisection achieved by auxiliary geometric construction.

\begin{figure}[H]
\centering
\includegraphics[width=0.5\textwidth]{12pt1.png}
\caption{Figure 12.1: Circle diagram showing angle \( \angle AOB \) trisected via Campanus method (c. 1250)}
\end{figure}

\newpage

\subsection*{Problem 4: Oresme’s Mean Velocity Rule (c. 1350 CE)}

\textbf{Statement.} Show that the distance under uniform acceleration equals the distance traveled under constant motion at average velocity.

\textbf{Solution.} A velocity-time triangle represents constant acceleration:
\[
\text{Area} = d = \frac{1}{2}tv = \left( \frac{v}{2} \right)t
\]
Thus, average velocity yields the same distance.

\textbf{Answer.} Merton Rule holds: distance equals average velocity times time.

\begin{figure}[H]
\centering
\includegraphics[width=0.5\textwidth]{12pt2.png}
\caption{Figure 12.2: Right triangle segmented by base; area equals mean velocity × time (c. 1350)}
\end{figure}

\begin{tcolorbox}[colback=gray!5!white,colframe=black!75!white,title=Scholastic Note: Infinity as Potential and Actual]

Medieval Scholastics inherited the paradoxes of infinity from Aristotle — but extended them with philosophical nuance.

\begin{itemize}
    \item \textbf{Infinity as Potential (\textit{infinitum potentiale}):}  
    A quantity is infinite if it can always be increased, but is never completed.  
    Example: The counting numbers — always one more to be added.

    \item \textbf{Infinity as Actual (\textit{infinitum actuale}):}  
    A quantity treated as a completed infinite totality.  
    Rarely accepted by strict Aristotelians, but subtly invoked by Oresme and others when reasoning about total change, infinite series, or divine attributes.
\end{itemize}

\textit{For most Schoolmen, infinity existed in God, but only potential in nature.}  
Oresme’s work nudged this boundary — treating infinite processes as geometrically summable, if not metaphysically complete.

\end{tcolorbox}


\newpage

\subsection*{Problem 5: Oresme’s Harmonic Series Divergence (c. 1360 CE)}

\textbf{Statement.} Show that the harmonic series diverges:
\[
\sum_{n=1}^{\infty} \frac{1}{n}
\]

\textbf{Solution.} Oresme grouped the terms:
\[
\left( \frac{1}{2} \right),\quad \left( \frac{1}{3} + \frac{1}{4} \right),\quad \left( \frac{1}{5} + \cdots + \frac{1}{8} \right), \dots
\]
Each group sums to more than \( \frac{1}{2} \), implying the series grows without bound.

\textbf{Answer.} The harmonic series diverges by lower bounding each group.

\begin{figure}[H]
\centering
\includegraphics[width=0.5\textwidth]{12extra6.png}
\caption{Oresme’s visual grouping proof for divergence of the harmonic series (c. 1360)}
\end{figure}

\newpage

\subsection*{Problem 6: Oresme’s “One and One-Half Proportion” Diagram (c. 1360 CE)}

\textbf{Statement.} In his \textit{Algorismus proportionum}, Nicole Oresme represented compounded proportions using geometric diagrams.  
One such figure depicts a square subdivided into labeled parts: \( p \), \( 1 \), \( 1 \), and \( 2 \), symbolizing the “one and one-half proportion.”

Explain how this square is constructed, what the labels mean, and how the overall expression corresponds to an early notion of irrational exponentiation.

\vspace{0.5em}

\textbf{Solution.}

Oresme begins with a unit square — the whole represents a **base quantity**, such as a velocity or intensity.

He then subdivides the square into **four regions**, each corresponding to a proportional component of an operation:

\begin{enumerate}
    \item The top-left square is labeled \( p \) — representing the base proportion.
    \item The top-right and bottom-left squares are labeled \( 1 \) — representing single proportions.
    \item The bottom-right is labeled \( 2 \) — representing a compound or enhanced proportion.
\end{enumerate}

This subdivision expresses a **compound proportion** built from combining a square root (half proportion) and a cube (triple proportion):

\[
\text{“One and one-half proportion”} = \left(\sqrt{x}\right)^3 = x^{3/2}
\]

Geometrically, the square represents the logical structure of:
- A base \( x \),
- Raised first to the \( 1/2 \) power (square root),
- Then cubed.

Oresme lacked modern notation, so he built conceptual understanding through spatial partitioning.

\vspace{0.5em}

\textbf{Answer.} The square decomposes the “one and one-half proportion” — or \( x^{3/2} \) — into labeled parts, visualizing the compounding of a square root and a cube via proportion geometry.

\begin{figure}[H]
\centering
\includegraphics[scale=0.2]{12extra5.png}
\caption{Oresme’s square diagram illustrating a “one and one-half proportion” or \( x^{3/2} \) (c. 1360)}
\end{figure}

\newpage

\section*{VI. Decline and Disruption: The Fall from the Scholastic High Point}

The Latin West reached a mathematical crescendo in the 13\textsuperscript{th} and 14\textsuperscript{th} centuries.  
Fibonacci systematized calculation.  
Jordanus and Campanus formalized number and geometry.  
Oresme introduced time, change, and infinitude into medieval diagrams.

But the surge was not sustained. What followed was not progress — but collapse.

\subsection*{I. The Black Death (1347–1351)}

The plague swept through Europe with devastating speed.  
It is estimated that up to 50\% of the population perished in less than five years.

\begin{itemize}
    \item Centers of learning — monastic libraries, cathedral schools, universities — lost faculty, scribes, and students.
    \item Mathematical manuscripts ceased to be copied; some were never replaced.
    \item Entire intellectual lineages vanished in a generation.
\end{itemize}

\subsection*{II. The Mongol Invasions (13\textsuperscript{th}–14\textsuperscript{th} c.)}

Though primarily felt in the East, the Mongol invasions disrupted the scholarly trade routes that had carried Arabic, Persian, and Indian mathematical works into Europe.

\begin{itemize}
    \item The destruction of Baghdad (1258) ended the Abbasid Caliphate and scattered scholars.
    \item The House of Wisdom, a cornerstone of transmission, was reduced to ash.
    \item The fragile bridges between Latin and Islamic scholarship were shaken.
\end{itemize}

\subsection*{III. Hundred Years’ War and Wars of the Roses (1337–1453, 1455–1487)}

In France and England — two intellectual powerhouses of the High Middle Ages — dynastic warfare consumed institutional resources and disrupted university life.

\begin{itemize}
    \item Oxford, Paris, and Cambridge suffered from shifting royal priorities and damaged endowments.
    \item Clerics and scholars were conscripted, displaced, or silenced.
    \item Investment in speculative learning waned in favor of theology and law.
\end{itemize}

\subsection*{IV. Institutional Inertia and Lingering Suspicion}

Even before the crisis, many clergy remained wary of Hindu-Arabic numerals and abstract reasoning.

\begin{itemize}
    \item Algorism competed with abacism for two centuries.
    \item Algebra remained mostly rhetorical — symbolic expression would not arise until the Renaissance.
    \item Mathematics, unlike theology or philosophy, lacked institutional champions.
\end{itemize}

\subsection*{Conclusion: A Flicker, Not a Flame}

Medieval mathematics did not end in disgrace — it ended in fragmentation.  
The high synthesis of Oresme and Fibonacci was not transmitted to apprentices — it was buried in libraries and eclipsed by plague, war, and politics.

It would take the Renaissance — and print — to dig these texts back up.

And when they were rediscovered, it would not be monks but merchants, artisans, and humanists who would relight the flame.


\newpage

\section*{VII. Closing Dialectic}

\textbf{Summary} \\

In the Latin West, mathematics was not discovered --- it was \textit{resurrected}. \\

From Roman collapse and monastic silence, a scattered inheritance was reassembled.

\begin{itemize}
    \item Greek definitions were not understood --- they were copied in faith.
    \item Arabic numerals were not trusted --- they were tested in trade.
    \item Algebra was not taught in academies --- it was smuggled into merchant ledgers.
    \item And infinity was not grasped --- it was sketched, hesitantly, by candlelight.
\end{itemize}

They gave us the Fibonacci sequence --- not as myth, but as model. \\

They gave us unit fractions --- not for beauty, but for bookkeeping. \\

They gave us coordinates --- not to map stars, but to measure change.

Latin mathematics was a mirror of its world: \\
Fragmented, faithful, practical --- a lattice of memory rather than a monument of theory.

\vspace{1em}

\textbf{Comparative Mathematical Cosmologies} \\

Greek: Number as essence. Mathematics as ideal. The infinite as paradox. \\

Islamic: Number as structure. Mathematics as inheritance. The infinite as solved. \\

Latin: Number as memory. Mathematics as survival. The infinite as suggestion.

\vspace{1em}

Each world did not merely receive mathematics ---  \\

It \textit{translated} it into the language of its needs, fears, and faith.

\vspace{1em}

\textbf{Exit Prompt} \\

You are Fibonacci. Or Jordanus. Or a silent monk in Reichenau with one scroll left to copy. \\

You do not know if what you write will be read.\\

But you write anyway.\\

No symbols, no printing press --- only ink, memory, and a vow. \\

What do you preserve? \\

What do you refuse to forget? \\

And who, centuries later, will know that you did?



\end{document}