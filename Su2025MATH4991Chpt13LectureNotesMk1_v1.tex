\documentclass[9pt]{article}
\usepackage{amsmath, amssymb, geometry, graphicx, float, fancyhdr}
\usepackage[most]{tcolorbox}

\geometry{margin=1in}
\pagestyle{fancy}
\fancyhf{}
\rhead{Su2025MATH4991 Lecture Notes}
\lhead{Chapter 13 - The European Renaissance}
\cfoot{\thepage}

\title{Chapter 13 -- The European Renaissance}
\author{Mr. Harley Caham Combest}
\date{June 2025}

\begin{document}
\maketitle

\section*{I. Cultural Invocation}

\begin{itemize}
    \item \textbf{Civilization:} Early Modern Europe (Italian City-States, German Principalities, French Kingdom, Tudor England, Spanish Habsburg Realms)
    \item \textbf{Time Period:} c. 1450--1600 AD
    \item \textbf{Figures:} 
    Regiomontanus (trigonometry, printing), \\
    Nicolas Chuquet (notation, exponents), \\
    Luca Pacioli (commercial arithmetic, bookkeeping), \\
    Adam Riese (Rechenmeister, abacus to algebra), \\
    Gerolamo Cardano (cubic solutions), \\
    Rafael Bombelli (imaginary numbers), \\
    Robert Recorde (equals sign), \\
    Fran\c{c}ois Vi\`ete (symbolic algebra, trigonometric solution of equations)
\end{itemize}

\begin{center}
    \includegraphics[scale=.2]{cover1.png}
\end{center}

\begin{quote}
\itshape
In the Renaissance, mathematics did not simply revive --- it \textbf{pivoted}.\\
From the ruins of plague and feudal inertia, a new language of number emerged: precise, symbolic, ambitious.\\

Gutenberg’s press cracked open the cloisters of learning. Merchants needed ledgers. Engineers needed angles. Artists needed perspective.\\
Into this maelstrom stepped mathematicians --- translators of Babylonian digits, Arabic algorithms, and Greek geometry into the nascent tongues of commerce, court, and craft.\\

Where the Latin West had \textit{survived}, the Renaissance now \textit{extended}: not merely copying but computing, not merely preserving but solving.\\
The triangle gained sine and tangent. The unknown gained letters. Equations gained cubic and quartic roots. And the page --- the printed page --- gained permanence.\\

This was not yet the age of calculus. But it was the age that made calculus \textbf{possible}.\\

Renaissance mathematics was not mystical, nor purely pragmatic.\\
It was transitional --- half-humanist, half-algebraist. A bridge of ink, angled through time.
\end{quote}



\newpage



\section*{II. Faces of the Era}

\textbf{Regiomontanus} (1436–1476)

\begin{itemize}
    \item Latin translator of Ptolemy's \textit{Almagest}; authored \textit{De Triangulis Omnimodis}, the first systematic text on trigonometry since antiquity.
    \item Established an observatory and one of the earliest mathematical printing presses in Nuremberg.
    \item Synthesized Islamic, Greek, and Latin mathematical traditions while pioneering a European revival of mathematical astronomy.
\end{itemize}

\textbf{Nicolas Chuquet} (1445–1488)

\begin{itemize}
    \item French physician and algebraist; wrote the \textit{Triparty en la science des nombres}, an advanced but unprinted work in symbolic algebra.
    \item Used exponential notation with zero and negative powers, foreshadowing later logarithmic developments.
    \item Recognized imaginary solutions and treated equations in a generalized algebraic structure centuries ahead of widespread adoption.
\end{itemize}

\textbf{Luca Pacioli} (1445–1514)

\begin{itemize}
    \item Italian Franciscan and author of the \textit{Summa de Arithmetica}; systematized commercial arithmetic and double-entry bookkeeping.
    \item Transmitted algebra and accounting to the rising mercantile class, paving the way for mathematical economics.
    \item Though not original in method, he was pivotal in dissemination; considered the "father of accounting."
\end{itemize}

\textbf{Adam Riese} (1492–1559)

\begin{itemize}
    \item German Rechenmeister; popularized Hindu-Arabic numerals and algorist methods over abacism.
    \item Authored widely used arithmetic texts and made the phrase “nach Adam Riese” synonymous with correct calculation.
\end{itemize}

\textbf{Gerolamo Cardano} (1501–1576)

\begin{itemize}
    \item Italian physician and polymath; published the \textit{Ars Magna}, revealing solutions to cubic and quartic equations.
    \item Engaged with negative and imaginary roots — calling them “sophistic” yet unavoidable.
    \item Sparked the transition from rhetorical to symbolic algebra and inspired algebraic exploration across Europe.
\end{itemize}

\textbf{Rafael Bombelli} (1526–1572)

\begin{itemize}
    \item Italian engineer; clarified operations with complex numbers in his \textit{Algebra}, recognizing the necessity of imaginary quantities in real solutions.
    \item First to conceptualize conjugate imaginaries as meaningful tools, not absurdities.
\end{itemize}

\textbf{Robert Recorde} (1510–1558)

\begin{itemize}
    \item Welsh physician and educator; introduced the equals sign ``$=$'' in \textit{The Whetstone of Witte} (1557).
    \item Wrote accessible English-language texts on arithmetic, algebra, astronomy, and geometry.
    \item Key figure in founding the English mathematical tradition.
\end{itemize}

\textbf{Fran\c{c}ois Vi\`ete} (1540–1603)

\begin{itemize}
    \item French cryptanalyst and mathematician; introduced systematic literal notation and parameters.
    \item Treated trigonometry as a symbolic analytic art; connected angle trisection to solving the irreducible cubic.
    \item Unified ancient geometry with emerging algebra — a precursor to Cartesian analytic geometry.
\end{itemize}



\newpage


\section*{III. Works of the Era}

\textbf{Regiomontanus — \textit{De Triangulis Omnimodis} (c. 1464, publ. 1533)}

\begin{itemize}
    \item First systematic work on planar and spherical trigonometry in Latin Europe.
    \item Proved the Law of Sines and solved general triangle problems using Euclidean geometry.
    \item Marked the rebirth of trigonometry as a discipline independent of astronomy.
\end{itemize}

\textbf{Regiomontanus — \textit{Tabulae Directionum} (c. 1475, publ. 1490)}

\begin{itemize}
    \item Contained sine and tangent tables calculated to high precision for astrological applications.
    \item Used large values for radius (``\textit{sinus totus}'') to avoid fractions — a common technique of the time.
    \item Early systematic treatment of the tangent function in European mathematics.
\end{itemize}

\textbf{Nicolas Chuquet — \textit{Triparty en la science des nombres} (1484, publ. 1880)}

\begin{itemize}
    \item Introduced a positional exponential notation: e.g., $6x^2$ as \texttt{.6.2}, and $x^{-2}$ as \texttt{.9.2.m}.
    \item Included arithmetic, root operations, and a pioneering algebra with negative and zero exponents.
    \item Anticipated logarithmic structures via tables of powers of 2.
\end{itemize}

\textbf{Luca Pacioli — \textit{Summa de Arithmetica, Geometria, Proportioni et Proportionalita} (1494)}

\begin{itemize}
    \item Encyclopedic text on arithmetic, algebra, geometry, and bookkeeping.
    \item Adopted symbolic shortcuts like \texttt{co} (cosa), \texttt{ce} (censo), and \texttt{ae} (aequalis).
    \item Declared cubic equations insoluble — a claim soon overtaken.
\end{itemize}

\textbf{Rafael Bombelli — \textit{L'Algebra} (1560, publ. 1572)}

\begin{itemize}
    \item Treated imaginary numbers as algebraically valid, using conjugate radicals to simplify real roots.
    \item Combined symbolic algebra with geometric interpretations (e.g., subdivision of cubes).
    \item Bridged Cardan’s formulas with new conceptual clarity on complex numbers.
\end{itemize}

\textbf{Gerolamo Cardano — \textit{Ars Magna} (1545)}

\begin{itemize}
    \item Published solutions to cubic and quartic equations; attributed quartic method to Ferrari.
    \item Employed substitutions like $x = u + v$ and completed the cube.
    \item Encountered complex roots while solving real equations — a watershed moment in algebra.
\end{itemize}

\textbf{Robert Recorde — \textit{The Whetstone of Witte} (1557)}

\begin{itemize}
    \item Introduced the equals sign ``$=$'' with the rationale: ``noe 2 thynges can be moare equalle.''
    \item Popularized algebra in English through rhetorical and practical teaching.
\end{itemize}

\textbf{Fran\c{c}ois Vi\`ete — \textit{Isagoge in Artem Analyticem} (1591)}

\begin{itemize}
    \item Introduced a distinction between knowns (consonants) and unknowns (vowels).
    \item Emphasized species-level algebra (logistica speciosa) over mere numbers (logistica numerosa).
    \item Unified symbolic expression with geometric reasoning; a precursor to modern algebraic formalism.
\end{itemize}

\textbf{Fran\c{c}ois Vi\`ete — \textit{Canon Mathematicus} (1579) and \textit{De Numerosa Potestatum... Resolutione} (1600)}

\begin{itemize}
    \item Computed high-precision trigonometric tables using decimal fractions.
    \item Advanced prosthaphaeresis: converting products into sums using trigonometric identities.
    \item Proposed an early iterative method for polynomial root approximation (akin to Horner’s method).
\end{itemize}



\newpage


\section*{IV. Historical Overview}

The European Renaissance did not invent mathematics anew — it reshaped and redistributed it.

\subsection*{1. Printing, Recovery, and Redundancy}

The invention of the printing press (c. 1440s) fundamentally altered the ecosystem of mathematical transmission:

\begin{itemize}
    \item Regiomontanus’s printing house in Nuremberg aimed to publish Greek, Arabic, and Latin classics — including Archimedes and Ptolemy.
    \item By 1500, over 30,000 editions of books had appeared; mathematics was a small subset, but now reproducible and portable.
    \item The printing press favored Latin over Greek, vernacular over elite notation — leading to parallel traditions in commercial and academic mathematics.
\end{itemize}

Though the fall of Constantinople (1453) is often credited with importing Greek texts to the West, its greater effect may have been symbolic: no longer could Byzantium be relied upon to preserve ancient memory. Western Europe had to build its own libraries — and its own mathematicians.

\subsection*{2. Algebra Reborn — From Abacists to Cossists}

The rebirth of algebra did not begin with proofs, but with merchants:

\begin{itemize}
    \item In France, Chuquet laid exponential and symbolic foundations.
    \item In Italy, Pacioli popularized commercial arithmetic and quadratic solutions.
    \item In Germany, Rechenmeisters like Riese and Rudolff compiled algebraic manuals called ``Coss'' books — from the Italian \textit{cosa} (thing).
    \item Symbols emerged piecemeal: p, m, co, ce, and eventually ``$=$'' in Recorde’s \textit{Whetstone}.
\end{itemize}

Algebra was not pure theory. It was bookkeeping, root extraction, inheritance splitting. The cossic tradition was algorithmic, not axiomatic — but from its algorithms, modern notation would crystallize.

\subsection*{3. Trigonometry: From Tables to Theory}

Regiomontanus’s \textit{De Triangulis} and Copernicus’s trigonometric supplements marked the rebirth of triangle-based reasoning in astronomy and geography.

\begin{itemize}
    \item Trigonometric tables were compiled to ten or more decimal digits by Rheticus and Otho.
    \item Prosthaphaeresis — converting products into sums — became a labor-saving tool in observatories.
    \item Vieta interpreted trigonometry algebraically, deriving angle-multiplication identities and using trigonometric substitutions to solve irreducible cubics.
\end{itemize}

From astronomy to surveying, trigonometry became a sovereign art.

\subsection*{4. Imaginary and Negative Numbers Enter the Discourse}

The Renaissance did not fully accept negative or imaginary numbers — but it could no longer avoid them.

\begin{itemize}
    \item Cardano encountered complex roots when solving real cubics — calling them ``sophistic.''
    \item Bombelli recognized that conjugate imaginaries could combine to yield real results — even if the square roots themselves were ``impossible.''
    \item Vieta and Harriot remained cautious, yet Girard began articulating the full consequences of admitting all roots — real, negative, and imaginary.
\end{itemize}

The tension between conceptual fidelity and computational necessity pushed algebra forward — even when metaphysics lagged behind.

\subsection*{5. Geometry, Perspective, and Projection}

While algebra and trigonometry advanced, geometry evolved in subtler ways:

\begin{itemize}
    \item Duurer and Alberti formalized methods of artistic perspective — linking vanishing points to geometric constructions.
    \item Werner and Maurolico revived conic sections and Pappian geometry.
    \item Mercator introduced cylindrical projection to preserve navigational direction — a fusion of geometry and geography.
\end{itemize}

The Renaissance opened geometry toward utility — even if the golden age of geometric theory would wait for Descartes.





\newpage



\section*{V. Problem–Solution Cycle}

\textbf{Problem 1: Werner’s Parabola by Tangent Circles (c. 1522)}

\textit{Statement.} Construct a parabola geometrically by using a pencil of circles tangent at a point on a horizontal axis.

\textit{Solution.}
\begin{itemize}
    \item Begin with a series of circles all tangent to point $a$ and centered successively along a horizontal axis at $b, c, d, e, f, g$.
    \item For each circle, construct vertical lines from its rightmost point, e.g., $C', D', E'$, and draw corresponding vertical lines at the same distances downward to $C'', D'', E''$.
    \item These reflected points define a parabolic arc, since each satisfies the condition:
    \[
        (\text{vertical segment})^2 = ab \cdot (\text{horizontal distance})
    \]
    \item This method approximates the parabola via compass-and-ruler constructions — a revival of ancient conic geometry.
\end{itemize}

\begin{center}
    \includegraphics[width=0.6\textwidth]{13pt1.png}

    \textit{FIG. 13.1: Werner’s geometric parabola via tangent circles}
\end{center}

\newpage

\textbf{Problem 2: Perspective Grid Construction (c. 1435, Alberti)}

\textit{Statement.} Using one vanishing point and two distance points, construct a realistic grid in linear perspective on a picture plane.

\textit{Solution.}
\begin{itemize}
    \item Let the horizon line intersect the vertical picture plane at point $V$ (vanishing point).
    \item On the groundline $RT$, choose equidistant divisions $A, B, \dots, G$.
    \item Connect $A$ through $G$ to $V$ — these become lines of convergence.
    \item Choose a distance point $P$ left of $V$ and connect it to each vertical point. Intersections define horizontal divisions of the projected trapezoids.
    \item This method constructs a perspectival tiling of squares with depth distortion — a method formalized in Alberti’s \textit{Della pittura}.
\end{itemize}

\begin{center}
    \includegraphics[width=0.8\textwidth]{13pt2.png}

    \textit{FIG. 13.2: Alberti’s perspective grid construction}
\end{center}

\newpage

\textbf{Problem 3: Approximate Nonagon Construction (c. 1525, Dürer)}

\textit{Statement.} Use intersecting arcs and proportional division to construct a regular 9-gon inscribed in a circle.

\textit{Solution.}
\begin{itemize}
    \item Start with an equilateral triangle inscribed in a circle: points $A, B, C$.
    \item Trisect the radius $AO$; label points $D$ and $E$ on the trisection.
    \item From center $O$, draw a smaller circle of radius $OE$.
    \item Intersect this smaller circle with the arcs determined by triangle vertices to get points $F, G$.
    \item The segment $FG$ approximates one side of a regular nonagon inscribed in the inner circle.
\end{itemize}

\begin{center}
    \includegraphics[width=0.4\textwidth]{13pt3.png}

    \textit{FIG. 13.3: Dürer’s approximate nonagon construction}
\end{center}

\newpage

\textbf{Problem 4: Viète’s Prosthaphaeretic Identity (c. 1593)}

\textit{Statement.} Prove the trigonometric identity:
\[
\sin x + \sin y = 2 \sin\left(\frac{x + y}{2}\right)\cos\left(\frac{x - y}{2}\right)
\]

\textit{Solution.}
\begin{itemize}
    \item Construct a circle with center $O$. Let $AB$ and $CD$ represent chords corresponding to angles $x$ and $y$.
    \item Join $A$ and $C$ across the diameter. Let $E$ lie at the base intersection of $AC$ extended.
    \item The chord sum $AB + CD = AE$ can be expressed as:
    \[
    \sin x + \sin y = AE = AC \cdot \cos\left(\frac{x - y}{2}\right)
    \]
    \item By angle bisection: $AC = 2 \sin\left(\frac{x + y}{2}\right)$
    \item Thus:
    \[
    \sin x + \sin y = 2 \sin\left(\frac{x + y}{2}\right)\cos\left(\frac{x - y}{2}\right)
    \]
    \item This was used by Viète in prosthaphaeresis to simplify multiplications into sum-difference forms — critical before logarithms.
\end{itemize}

\begin{center}
    \includegraphics[width=0.5\textwidth]{13pt4.png}

    \textit{FIG. 13.4: Viète’s geometric derivation of the prosthaphaeretic identity}
\end{center}


\newpage

\section*{VI. Decline and Disruption: The Limits of Renaissance Form}

The Renaissance reawakened mathematical ambition — but it did not yet achieve mathematical integration.

\subsection*{I. The Humanist Constraint}

The Humanists prized elegance, rhetoric, and the recovery of Greek ideals — but they often resisted technical symbolism:

\begin{itemize}
    \item Chuquet’s exponential notation was buried for centuries.
    \item Recorde’s equals sign ``$=$'' remained unused in Continental texts for generations.
    \item Algebra remained largely rhetorical — filled with \textit{res}, \textit{cosa}, and verbose formulations rather than symbolic abstraction.
\end{itemize}

The very classicism that elevated Renaissance thought also slowed its transformation into modern symbolic mathematics.

\subsection*{II. Incomplete Absorption of Greek Depth}

Though translations of Archimedes, Apollonius, and Pappus entered circulation:

\begin{itemize}
    \item Their deeper geometric content was known to only a few (e.g., Maurolico, Commandino).
    \item Algebra and trigonometry advanced without full integration into classical synthetic geometry.
    \item Conics were revived, but analytic geometry had not yet been conceived.
\end{itemize}

The ancient tools were recovered — but not yet retooled.

\subsection*{III. Imaginary and Negative Numbers Remained Suspicious}

Despite Bombelli’s clarity and Cardano’s accidental discoveries:

\begin{itemize}
    \item Most algebraists treated negative roots as absurd or inadmissible.
    \item Imaginary numbers were used functionally but not conceptually accepted.
    \item The full number line — from $-\infty$ to $+\infty$, through the complex plane — had not yet been philosophically or pedagogically embraced.
\end{itemize}

They could solve what they could not yet believe in.

\subsection*{IV. Fragmentation Across Regions and Languages}

Europe's mathematical culture was vibrant but decentralized:

\begin{itemize}
    \item German Cossists, Italian algebraists, English educators, and French symbolists worked in parallel.
    \item Notation was inconsistent — $p/m$ in Italy, $+/-$ in Germany, verbal in England, and mixed in France.
    \item Nationalism and confessional divides (Reformation and Counter-Reformation) further fractured mathematical discourse.
\end{itemize}

The Renaissance had awakened the mathematical faculties — but not yet unified them.

\subsection*{V. Geometry Lacked Direction Beyond Perspective}

While perspective, cartography, and polyhedral aesthetics flourished:

\begin{itemize}
    \item Pure geometry stalled in procedural approximations (e.g., Dürer’s nonagon).
    \item No coordinate system linked geometric curves to algebraic functions.
    \item The projective depth of conics, loci, and constructibility remained scattered across disciplines.
\end{itemize}

The Renaissance had depth — but not yet dimensions.

\subsection*{Conclusion: Awaiting the Cartesian Threshold}

By 1600:

\begin{itemize}
    \item Algebra had become more general — but not yet fully symbolic.
    \item Geometry had become more applied — but not yet analytic.
    \item Trigonometry had become more precise — but not yet functional.
\end{itemize}

The Renaissance had rediscovered tools.  
The seventeenth century would begin to wield them.




\newpage

\section*{VII. Closing Dialectic}

\textbf{Summary}

In the European Renaissance, mathematics did not merely return —  
It was reframed.

\begin{itemize}
    \item Greek geometry was studied — but through Latin lenses and Italian margins.
    \item Arabic algebra was used — but without always naming its sources.
    \item Hindu-Arabic numerals were taught — but resisted for generations.
    \item The cubic was solved — but its complex roots remained mistrusted.
    \item Perspective was drawn — but not yet calculated.
\end{itemize}

The Renaissance was not the final synthesis — it was the tremor before Descartes.\\

It gave us Regiomontanus, who printed the triangle.  
Chuquet, who named the power.  
Pacioli, who trained the merchant.  
Cardano, who violated the oath.  
Bombelli, who embraced the impossible.  
Recorde, who gave us “$=$”.  
Viète, who distinguished the known from the unknown.\\

Their works were fragmented.  
Their language inconsistent.  
Their belief incomplete.\\

And yet —\\

They moved the unknown into ink.  
They gave form to what could not yet be believed.  
They stood at the edge of the symbolic and looked across.

\textbf{Comparative Mathematical Cosmologies}

\begin{itemize}
    \item \textbf{Greek:} Mathematics as essence. Geometry as argument. Infinity as paradox.
    \item \textbf{Islamic:} Mathematics as inheritance. Algebra as method. Infinity as solved.
    \item \textbf{Renaissance:} Mathematics as recovery. Number as symbol. Infinity as suggestion.
\end{itemize}

Each did not merely solve — they remembered, reframed, and reprojected.\\

\textbf{Exit Prompt}\\

You are Viète. Or Recorde. Or an anonymous Rechenmeister copying powers in candlelight.\\

You do not yet know the Cartesian plane.  
You have never seen a logarithm.  
You mistrust negatives.  
You do not believe in $i$.\\

But you solve anyway.  
You write anyway.  
You define, calculate, and preserve.\\

Not because you have clarity —  
But because you have \textit{memory and vow}.\\

\textbf{What symbol do you preserve?}  
\textbf{What notation do you risk?}  
\textbf{And who, four centuries later, will know that you did?}


\end{document}