\documentclass[9pt]{article}
\usepackage{amsmath, amssymb, geometry, graphicx}
\usepackage{titlesec}
\titleformat{\section}[block]{\large\bfseries}{\thesection}{1em}{}
\titleformat{\subsection}[runin]{\bfseries}{}{0pt}{}[.]

\begin{document}

\begin{center}
\Large\textbf{Chapter 18 – Pre and Post Revolutionary France} \\
\large Harley Caham Combest \\
\large Su2025 MATH4991 Lecture Notes – Mk1
\end{center}

\vspace{1em}

\section*{I. Cultural Invocation}

\begin{itemize}
  \item \textbf{Civilization:} France in Revolutionary Convulsion --- Enlightenment’s apex meets political rupture.
  \item \textbf{Time Period:} c.\ 1770--1830
  \item \textbf{Epochs:} Fall of the Ancien Régime; Rise of the Metric State; Birth of the \textit{École Polytechnique}; Napoleonic Technocracy; Restoration and Diffusion
  \item \textbf{Figures:} Jean d’Alembert, Gaspard Monge, Joseph-Louis Lagrange, Pierre-Simon Laplace, Adrien-Marie Legendre, Lazare Carnot, Nicolas de Condorcet
\end{itemize}

\noindent
Mathematics in France became more than inquiry --- it became instrument of governmental thought and expansion.\\

\noindent
The collapse of monarchy and the ascent of revolutionary reason did not diminish mathematical power; they re-forged it. \\

\noindent
Education was weaponized. Geometry instructed artillery. Number governed the meter. \\

\noindent
Mathematicians became architects of state, theorists of structure, and voices of reform.\\

\noindent
This was not a return to classical purity, nor an abandonment to abstraction. It was the forging of \textit{rigorous modernity} under revolutionary stress --- where pedagogy, policy, and principle converged.\\


\vspace{1em}

\begin{center}
    \includegraphics[scale=.31]{PreToPostRevFrance.png}\\
    \textit{Mathematics became the grammar of the power.}
\end{center}

\newpage

\section*{II. Big Pictures}

\begin{center}
    \includegraphics[scale=.15]{cover2.png}\\
    \vspace{1em}
    \textit{1812, before The Push into Russia}
\end{center}

\newpage

\section*{III. Faces of the Era}

\subsection*{Jean le Rond d’Alembert (1717–1783)}
\begin{itemize}
  \item Philosophe, encyclopedist, and architect of analytic rigor. He wrote much of the \textit{Encyclopédie}, including its mathematical and scientific content.
  \item Opposed the use of infinitesimals on metaphysical grounds; favored a proto-limit approach rooted in clarity and logic.
  \item Derived the wave equation for vibrating strings and proposed the Cauchy-Riemann equations in early form.
  \item Believed that the calculus should be founded on limits, not on infinitely small quantities: “A quantity is either something or nothing.”
  \item His principle: differentiation is the limit of finite differences — an early articulation of derivative-as-process.
\end{itemize}

\subsection*{Étienne Bézout (1730–1783)}
\begin{itemize}
  \item Algebraist and educator. Authored military mathematics textbooks influential at institutions like Mézières.
  \item Developed elimination theory using determinants, now bearing his name in the “Bézoutian” and “Bézout’s Theorem.”
  \item First to systematically treat the intersection of algebraic curves by degree.
  \item His \textit{Cours de mathématiques} was widely used in both France and the early United States.
\end{itemize}

\subsection*{Marie Jean Antoine Nicolas de Caritat, Marquis de Condorcet (1743–1794)}
\begin{itemize}
  \item Philosophe and advocate of human progress. A mathematician of probability and social welfare.
  \item Wrote on integral calculus, education, statistics, and political decision-making.
  \item Proposed one of the first formal models of voting theory — the Condorcet method.
  \item Composed the visionary \textit{Sketch for a Historical Picture of the Progress of the Human Mind} while in hiding during the Revolution.
  \item Died in prison under unclear circumstances — the only one of the six to perish by revolutionary fallout.
\end{itemize}

\subsection*{Joseph-Louis Lagrange (1736–1813)}
\begin{itemize}
  \item Author of the \textit{Mécanique analytique}; sought to rebuild calculus on purely algebraic grounds.
  \item Eliminated geometric intuition and infinitesimals in favor of “derived functions” — laying early foundations for function theory.
  \item Inventor of the calculus of variations; contributed to group theory, number theory, and algebraic solvability.
  \item Formulated the method of Lagrange multipliers and clarified polynomial permutation theory.
  \item Bridged Enlightenment elegance with post-revolutionary rigor.
\end{itemize}

\subsection*{Gaspard Monge (1746–1818)}
\begin{itemize}
  \item Father of descriptive geometry and founder of the École Polytechnique.
  \item Elevated geometry to national service: perspective, shadow, topography, and engineering.
  \item Taught the first systematic courses in solid and differential geometry.
  \item Authored the \textit{Géométrie descriptive} and advanced analytic geometry in three dimensions.
  \item Helped redefine the relationship between geometric representation and analytic method.
\end{itemize}

\subsection*{Lazare Carnot (1753–1823)}
\begin{itemize}
  \item Military organizer and metaphysician of the infinitesimal.
  \item His \textit{Réflexions sur la métaphysique du calcul infinitésimal} offered a principle of compensating errors as the foundation of calculus.
  \item Extended Menelaus’s and Heron’s theorems; proposed intrinsic coordinates including aberrancy.
  \item Introduced the idea of “geometry of position” and sought a unification of geometric theorems through algebraic correlation.
  \item Mathematician, general, and revolutionary statesman — a rare fusion of logic and leadership.
\end{itemize}

\subsection*{Pierre-Simon Laplace (1749–1827)}
\begin{itemize}
  \item Author of the \textit{Mécanique céleste}; completed Newton’s vision using analytic techniques.
  \item Developed the theory of probability, including Bayes' theorem and the Poisson distribution.
  \item Introduced the potential function, Laplace's equation, and operator theory.
  \item Advocated determinism in mechanics and dismissed divine intervention as unnecessary in physical explanation.
  \item Merged astronomy, physics, and mathematics into one coherent analytic cosmos.
\end{itemize}

\subsection*{Adrien-Marie Legendre (1752–1833)}
\begin{itemize}
  \item Brought rigor to geometry in his \textit{Éléments de géométrie}.
  \item Published major work on elliptic integrals, Legendre polynomials, and beta/gamma functions.
  \item Early contributor to number theory: quadratic reciprocity, Legendre symbol, and distribution of primes.
  \item Helped standardize methods in geodesy and data analysis — including least squares estimation.
  \item Quiet reformer of rigor across analysis and arithmetic.
\end{itemize}

\subsection*{Joseph Fourier (1768–1830)}
\begin{itemize}
  \item Synthesized analysis and heat theory in the \textit{Théorie analytique de la chaleur}.
  \item Introduced Fourier series — allowing representation of arbitrary functions via trigonometric expansions.
  \item Expanded the definition of function and opened the door to modern analysis.
  \item Influenced by and later mentored Dirichlet, Sturm, and others in early 19th-century Paris.
  \item His work formed the bridge from Enlightenment physics to functional analysis.
\end{itemize}

\subsection*{Augustin-Louis Cauchy (1789–1857)}
\begin{itemize}
  \item Systematizer of the calculus. Formalized definitions of limit, continuity, derivative, and convergence.
  \item Pioneered complex function theory, determinants, Cauchy integrals, and uniform convergence.
  \item Authored textbooks that shaped rigorous analysis across Europe.
  \item Introduced criteria for convergence and methods for solving differential equations with existence theorems.
  \item More than a reformer — he engraved rigor into the bones of modern mathematics.
\end{itemize}

\subsection*{Siméon Denis Poisson (1781–1840)}
\begin{itemize}
  \item Physicist and mathematician who bridged the analytical heritage of Laplace with 19th-century physics.
  \item Contributed to potential theory, probability, celestial mechanics, elasticity, and electrostatics.
  \item Known for: Poisson integral, Poisson brackets, Poisson’s equation, Poisson’s ratio, and the Poisson distribution.
  \item Authored nearly 400 publications, including influential treatises on mechanics and probability theory.
  \item Though less foundational in pure theory, he preserved and extended the Napoleonic analytical tradition in applied form.
\end{itemize}

\vspace{1em}

At this point the number of works and their names has sky rocketed; so, let us turn instead to...

\newpage

\section*{IV. Institutions and Reforms}

\subsection*{Académie des Sciences and the Institut National}
\begin{itemize}
  \item The Royal Academy of Sciences, France’s leading mathematical institution throughout the 18th century, was shut down by the Revolution in 1793.
  \item In 1795, the Institut National des Sciences et des Arts was created to replace the former academies.
  \item It was organized into three classes: physical and mathematical sciences, moral and political sciences, and literature and the fine arts.
  \item Napoleon joined the Institut in 1797, elevating its political and scientific prestige.
  \item Though reorganized multiple times, the Academy’s mathematical activity was largely preserved under various names.
\end{itemize}

\subsection*{The Committee on Weights and Measures}
\begin{itemize}
  \item Tasked with the rationalization of measurement standards, the committee included Lagrange, Laplace, Monge, and Legendre.
  \item Debated standardization via the pendulum versus terrestrial measurements.
  \item Ultimately defined the meter as one ten-millionth of the distance from the equator to the North Pole.
  \item The metric system was finalized by 1799 and became a lasting mathematical legacy of the Revolution.
  \item Though secondary to theoretical advances, this reform gave mathematics tangible administrative power.
\end{itemize}

\subsection*{École Normale (1794)}
\begin{itemize}
  \item Hastily established to train teachers and rebuild France’s collapsed educational system.
  \item Hosted lectures by Monge, Lagrange, Laplace, and Legendre.
  \item Though short-lived in its original form, it became a precedent for state-supported mathematical instruction.
  \item Lecture notes from this school, especially Lagrange’s and Monge’s, influenced pedagogy for decades.
\end{itemize}

\subsection*{École Polytechnique (Founded 1794)}
\begin{itemize}
  \item The crown jewel of revolutionary mathematical education.
  \item Founded with strong input from Monge, Carnot, and others to train engineers, artillerymen, and state scientists.
  \item Monge served as both administrator and lead instructor, shaping the geometry and analysis curriculum.
  \item Students were taught descriptive geometry, analytic geometry in three dimensions, mechanics, and applied calculus.
  \item Became the prototype of a modern scientific-military academy — its influence spread across Europe and America.
\end{itemize}

\subsection*{Educational Diffusion and the Textbook Revolution}
\begin{itemize}
  \item The École Polytechnique curriculum spawned a wave of analytic geometry textbooks.
  \item Authors included Biot, Lacroix, Puissant, and Lefrançois — all building on Monge’s lectures.
  \item Lacroix’s works reached over twenty editions and were translated into English for American use.
  \item The term “analytic geometry” came into popular use through these texts.
  \item Mathematics became structured, hierarchical, and teachable at scale — a legacy still felt today.
\end{itemize}

\newpage

\section*{V. Domains and Developments}

\subsection*{Descriptive and Analytic Geometry}
\begin{itemize}
  \item Monge’s \textit{Géométrie descriptive} introduced the method of double orthographic projection — used for military design and engineering.
  \item His lectures at the École Polytechnique formalized solid analytic geometry and differential geometry.
  \item Analytic geometry in three dimensions became central: coordinates, tangent planes, intersection curves, and transformation of axes.
  \item The Monge point, Monge circle, and director sphere emerged from his work in tetrahedral geometry.
  \item Textbooks proliferated, pushing solid geometry into the mathematical mainstream for the first time since antiquity.
\end{itemize}

\subsection*{Calculus of Variations}
\begin{itemize}
  \item Lagrange pioneered the subject, formalizing it with his method of undetermined multipliers and notation of variation.
  \item Euler delayed his own publication to allow Lagrange full credit for his general techniques.
  \item Variational problems such as brachistochrones, geodesics, and isoperimetric curves became central examples in analysis.
  \item The subject deepened the relationship between geometry, mechanics, and optimization — foreshadowing functional analysis.
\end{itemize}

\subsection*{Foundations of Analysis}
\begin{itemize}
  \item Lagrange attempted to found calculus purely on algebraic expansions — leading to the theory of derived functions.
  \item D’Alembert and Carnot both pursued alternative metaphysical foundations for calculus: limits and compensation of errors, respectively.
  \item Cauchy established the formal limit definition of continuity, derivative, and convergence — giving rise to modern analysis.
  \item The limit displaced the infinitesimal as the foundation of rigor.
  \item These developments marked the dawn of the \textit{arithmetization of analysis}.
\end{itemize}

\subsection*{Function Theory}
\begin{itemize}
  \item Lagrange’s \textit{Théorie des fonctions analytiques} sought to eliminate infinitesimals by defining derivatives via series.
  \item Fourier expanded the concept of function beyond differentiability with trigonometric series — enabling arbitrary periodic representation.
  \item Dirichlet later defined functions as arbitrary correspondences between variables, vastly expanding the analytic landscape.
  \item Cauchy developed the theory of complex functions and introduced key theorems of analytic continuation and contour integration.
\end{itemize}

\subsection*{Probability and Statistics}
\begin{itemize}
  \item Laplace systematized probability theory in his \textit{Théorie analytique des probabilités} and introduced the Laplace transform.
  \item Condorcet applied probability to voting and decision-making — laying the groundwork for social choice theory.
  \item Legendre introduced the method of least squares; Laplace later formalized its probabilistic foundation.
  \item Poisson developed discrete distributions, including the Poisson distribution, and extended probability to judicial applications.
\end{itemize}

\subsection*{Differential Equations and Mechanics}
\begin{itemize}
  \item D’Alembert derived and solved wave equations; Monge and Lagrange applied differential equations to motion and surfaces.
  \item Laplace's operator and potential theory became foundational in celestial mechanics, acoustics, and gravitation.
  \item Cauchy introduced methods for solving both ordinary and partial differential equations — including convergence guarantees.
  \item Lagrange’s variation of parameters, Cauchy’s majorants, and the Cauchy-Kowalewski theorem advanced existence theory.
\end{itemize}

\subsection*{Number Theory and Algebra}
\begin{itemize}
  \item Legendre authored the first major treatise on number theory and proved special cases of Fermat’s Last Theorem.
  \item Lagrange proved the four-square theorem and early results on congruences; laid groundwork for group theory.
  \item The law of quadratic reciprocity, rediscovered by Legendre, became central to modular arithmetic.
  \item Bézout developed elimination theory and algebraic methods anticipating determinant theory.
\end{itemize}

\newpage

\section*{VI. Canonical Problems and Solutions}

\subsection*{Problem 1: What is a rational definition of length? (Metric Reform, 1790–1799)}
\begin{itemize}
  \item The Committee on Weights and Measures considered defining the meter via pendulum length or the Earth’s meridian.
  \item Ultimately adopted: the meter is \(\frac{1}{10{,}000{,}000}\) of the distance from equator to pole along a meridian.
  \item This decision prioritized Earth-based measurement over oscillatory definitions, enabling international standardization.
  \item Result: A decimal-based metric system, still in global use — the most tangible mathematical artifact of the French Revolution.
\end{itemize}

\subsection*{Problem 2: Can arbitrary functions be expressed as trigonometric series? (Fourier, 1822)}
\begin{itemize}
  \item Fourier proposed that any function \( f(x) \) defined on an interval could be written as a sum of sines and cosines.
  \item The Fourier series:
  \[
  f(x) = \frac{a_0}{2} + \sum_{n=1}^\infty \left( a_n \cos nx + b_n \sin nx \right)
  \]
  \item Coefficients derived by integrals over the interval — valid even for piecewise and non-differentiable functions.
  \item Introduced ideas of convergence, pointwise behavior, and functional generality.
  \item Led to the creation of functional analysis and redefinition of “function” itself.
\end{itemize}

\subsection*{Problem 3: How can constraints be enforced in optimization? (Lagrange Multipliers)}
\begin{itemize}
  \item Given a function \( f(x, y, z) \) with constraints \( g(x, y, z) = 0 \), Lagrange introduced a system of augmented equations.
  \item Define:
  \[
  \mathcal{L}(x, y, z, \lambda) = f(x, y, z) + \lambda g(x, y, z)
  \]
  \item Solve \( \nabla \mathcal{L} = 0 \) to find critical points satisfying the constraint.
  \item This method remains a cornerstone in modern optimization, economics, and physics.
\end{itemize}

\subsection*{Problem 4: Are all polynomial equations solvable? (Lagrange, 1770)}
\begin{itemize}
  \item Lagrange studied permutations of roots and resolvents to assess solvability.
  \item Discovered that general equations of degree \( >4 \) cannot be solved by radicals.
  \item This led to the development of group theory and Galois theory in the 19th century.
  \item Conjectured what Abel and Galois would later prove — the quintic is generally unsolvable.
\end{itemize}

\subsection*{Problem 5: How can algebraic curves be eliminated? (Bézout, 1779)}
\begin{itemize}
  \item Bézout devised systematic use of determinants to eliminate variables from systems of polynomial equations.
  \item Developed the Bézoutian matrix and established his theorem: two algebraic curves of degrees \( m \) and \( n \) intersect in \( mn \) points (in general).
  \item Created a foundation for algebraic geometry and projective elimination theory.
\end{itemize}

\subsection*{Problem 6: Can differentials be defined without infinitesimals? (D’Alembert, Cauchy)}
\begin{itemize}
  \item D’Alembert: defined derivatives via limits of finite difference quotients.
  \item Cauchy: formalized this with the modern limit:
  \[
  f'(x) = \lim_{h \to 0} \frac{f(x+h) - f(x)}{h}
  \]
  \item Replaced intuitive “infinitesimals” with rigorously defined limits.
  \item Set the standard for 19th-century analysis and the future of calculus instruction.
\end{itemize}

\subsection*{Problem 7: What is the probability of rare events? (Poisson Distribution)}
\begin{itemize}
  \item Poisson studied discrete random events with low probability and high frequency of trials.
  \item Found the limiting case of the binomial distribution:
  \[
  P(k; \lambda) = \frac{\lambda^k e^{-\lambda}}{k!}
  \]
  \item Applicable to counting occurrences in time or space: arrivals, defects, mutations.
  \item Foundational for statistical physics and modern probability modeling.
\end{itemize}

\subsection*{Problem 8: How can the Earth’s gravitational attraction be modeled? (Legendre, Laplace)}
\begin{itemize}
  \item Legendre derived zonal harmonics to model gravitational fields of ellipsoids.
  \item Laplace introduced the potential function \( \Phi \) and the Laplace operator \( \nabla^2 \Phi = 0 \).
  \item Their work unified celestial mechanics, fluid dynamics, and electrostatics under a common analytic framework.
  \item Provided mathematical foundation for geodesy, astronomy, and field theory.
\end{itemize}


\newpage

\section*{VII. Closing Dialectic}

The French Revolution reshaped mathematics not merely by content, but by context. Institutions collapsed — and from their ruins emerged the first state-engineered curriculum for mathematical power. Geometry became tactical. Analysis became systemic. The École Polytechnique, with Monge as its architect and Lagrange as its high priest, became the axis around which the next half-century of instruction would rotate.

For a time, Paris stood unmatched:  
\begin{itemize}
  \item Laplace rewrote the heavens in equations.  
  \item Lagrange reconsecrated mechanics in symbols.  
  \item Cauchy made rigor a virtue.  
  \item Fourier gave irregularity a voice.  
  \item Legendre and Poisson refined number, probability, and potential into tools.  
\end{itemize}

But empires of thought, like empires of state, do not endure without renewal.

\subsection*{Why the French Ascendancy Waned}
\begin{itemize}
  \item \textbf{Educational ossification:} The École Polytechnique became doctrinal — its lectures memorized, its innovation dulled. The very formalism it birthed began to constrain.
  \item \textbf{Philosophical insularity:} Where French mathematics aimed at completeness and clarity, German mathematics turned toward abstraction and infinitude.
  \item \textbf{Revolution’s aftermath:} While France struggled with waves of political upheaval after 1830, Prussia and England invested in long-term academic reform.
  \item \textbf{Lack of successors:} After Cauchy, few French mathematicians matched the originality of their predecessors. Meanwhile, Berlin and Cambridge cultivated a new generation of theorists.
\end{itemize}

\subsection*{Rise of the Successors}
\begin{itemize}
  \item \textbf{England:} The Analytical Society (Peacock, Herschel, Babbage) broke free from Newtonian isolation, reimported Continental rigor, and laid groundwork for symbolic algebra and computation.
  \item \textbf{Prussia:} Gauss, Dirichlet, Jacobi, and Riemann extended and surpassed the analytic groundwork laid in Paris. Berlin became the new capital of abstraction — function theory, number theory, and geometry reimagined.
  \item \textbf{Journals and Institutions:} While France had the École, Germany built a research university model — flexible, diffuse, and generationally sustainable.
\end{itemize}

\vspace{1em}

\begin{center}
Next page is Final Chpt18 Page $=>$
\end{center}

\subsection{Closing Statement}

\begin{center}
  \textit{The Revolution gave mathematics new instruments. But it was the next century that learned to play them.}
\end{center}

\vspace{1em}

\noindent
France in the revolutionary era transformed mathematics from a guild to a grammar — from a scholar’s pursuit to a state language. \\

\noindent
But it was in Göttingen and Königsberg, in Cambridge and Berlin, that mathematics became a philosophy again.



\end{document}