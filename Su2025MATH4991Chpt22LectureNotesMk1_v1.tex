\documentclass[9pt]{article}
\usepackage{amsmath, amssymb, geometry, graphicx}
\usepackage{titlesec}
\titleformat{\section}[block]{\large\bfseries}{\thesection}{1em}{}
\titleformat{\subsection}[runin]{\bfseries}{}{0pt}{}[.]

\begin{document}

\begin{center}
\Large\textbf{Chapter 22 – Analysis} \\
\large Harley Caham Combest \\
\large Su2025 MATH4991 Lecture Notes – Mk1
\end{center}

\vspace{1em}

\section*{I. Cultural Invocation}

\begin{itemize}
  \item \textbf{Civilization:} 19th-century Europe — post-Newtonian, post-Eulerian, striving for mathematical rigor amid physical discovery.
  \item \textbf{Time Span:} Early 1800s to early 1900s
  \item \textbf{Epochal Axes:} Continuity, Convergence, Integration, Real Number Construction, Function Theory
  \item \textbf{Figures:} Cauchy, Riemann, Weierstrass, Dedekind, Cantor, Heine, Dirichlet, Bolzano, Hermite, Liouville, Poincaré
\end{itemize}

\noindent
Analysis is the mathematics of change —  
but in the 19th century, it became the mathematics of control.\\

\noindent
After centuries of intuitive manipulation — Newtonian fluents, Eulerian expansions — cracks appeared.  
Infinitesimals had no grounding. Convergence was misunderstood. Continuity was presumed.\\

Rigor had to be rebuilt.

\begin{itemize}
  \item Cauchy redefined limits and derivatives through clarity, not speed.
  \item Weierstrass formalized $\varepsilon$–$\delta$ language and banished intuition.
  \item Riemann, from intuition again, created integrals, surfaces, and the zeta function.
  \item Dedekind cut the rationals to construct the reals.
  \item Cantor made infinity mathematical — and countability a property.
\end{itemize}

Each contribution was both surgical and philosophical:\\

\noindent
- Analysis was not just about solving problems — it was about defining the conditions under which solutions mean anything.\\

\noindent
- Where algebra built structures, analysis secured the foundations beneath them.\\

It was not a linear refinement.  
It was a reconstruction — brick by logical brick — of what “real” even means.

\newpage

\section*{II. Big Pictures}

\noindent
Analysis was not simply extended in the 19th century — it was redefined.\\

\noindent
Calculus had worked. But its tools were heuristic, its foundations unclear.  \\

\noindent
The 1800s did not abandon its power — they rebuilt its logic.\\

\noindent
Here are the major arcs that shaped this reconstruction:

\bigskip

\textbf{1. Cauchy’s Reformation}  
- Defined limits and continuity with rigor.  
- Introduced power series and formal convergence.  
- Founded rigorous function theory and complex analysis.

\bigskip

\textbf{2. Weierstrass and the $\varepsilon$–$\delta$ Revolution}  \\

- Banished infinitesimals from analysis.  \\

- Defined limits, continuity, and differentiability with formal logic.  \\

- Made pathological functions central — continuous nowhere-differentiable examples.

\bigskip

\textbf{3. Riemann’s Intuition}  \\

- Reclaimed intuition with depth: integration, surfaces, analytic continuation.  \\

- Introduced the Riemann integral, Riemann surfaces, and the Riemann zeta function.  \\

- Bridged physics, geometry, and function theory.

\bigskip

\textbf{4. Construction of the Real Numbers}  \\

- Dedekind cuts and Cantor sequences made the real line logically complete. \\

- Méray, Heine, Bolzano contributed to convergence, series, and irrationality.  \\

- Arithmetic replaced geometry as the foundation of analysis.

\bigskip

\textbf{5. Cantor’s Set Theory and Infinity}  \\

- Countability, uncountability, and transfinite numbers.\\

- Function spaces became sets with structure — not just rules.  \\

- Real analysis became inseparable from set theory.

\bigskip

\textbf{6. National Currents} \\

- \textit{Germany}: Rigor, completeness, function theory (Weierstrass, Dirichlet, Dedekind, Cantor).  \\

- \textit{England}: Physics-led analysis (Stokes, Maxwell, Airy).  \\

- \textit{France}: Analytical continuity and transcendence (Cauchy, Hermite, Poincaré).

\bigskip

\noindent
Together these arcs did more than restore clarity.  
They redrew what analysis was — and what it was for.

\vspace{1em}

\noindent
From motion to logic. From curves to completeness.  
Analysis ceased to describe the world, and began to define its mathematics.

\newpage

\section*{III. Epochal Outline — Analysis Through Iteration}

\noindent
Analysis matured through confrontation.\\

\noindent
Each generation found flaws in the last — and redefined what it meant to be correct, convergent, or complete.\\

\noindent
Here are eight major phases in the transformation of analysis:

\bigskip

\textbf{1. Pre-Rigorous Calculus} \hfill \textit{(Newton, Euler)}

\begin{itemize}
  \item Derivatives and integrals treated as operations on infinitesimals.
  \item Formal expansions used freely, often diverging or misapplied.
  \item Power without precision — results were right, reasoning uncertain.
\end{itemize}

\bigskip

\textbf{2. Cauchyan Function Theory}

\begin{itemize}
  \item Limits, derivatives, and continuity defined via sequences and convergence.
  \item The derivative becomes a local property defined by a limit.
  \item Complex analysis begins in earnest.
\end{itemize}

\bigskip

\textbf{3. Weierstrassian Rigidity}

\begin{itemize}
  \item Analysis becomes fully formal — $\varepsilon$–$\delta$ definitions rule.
  \item Continuity, differentiability, and convergence are divorced from intuition.
  \item Pathological functions (continuous but nowhere differentiable) are accepted, even embraced.
\end{itemize}

\bigskip

\textbf{4. Riemannian Integration and Surfaces}

\begin{itemize}
  \item Defines integration as a limiting process over partitions.
  \item Introduces multi-valued functions, surfaces, and analytic continuation.
  \item Analysis becomes geometric again — but abstractly.
\end{itemize}

\bigskip

\textbf{5. Construction of the Reals} \hfill \textit{(Dedekind, Cantor, Méray)}

\begin{itemize}
  \item The real number line is constructed from rational data.
  \item Continuity becomes a property of completeness — not intuition.
  \item The irrational and transcendental gain rigorous meaning.
\end{itemize}

\bigskip

\textbf{6. Set Theory Enters} \hfill \textit{(Cantor)}

\begin{itemize}
  \item Infinite sets are classified: countable vs. uncountable.
  \item Functions are reconceived as set-theoretic mappings.
  \item Transfinite numbers extend arithmetic beyond the finite.
\end{itemize}

\bigskip

\textbf{7. Analysis and Physics} \hfill \textit{(Stokes, Kelvin, Maxwell)}

\begin{itemize}
  \item Analysis is applied to wave equations, heat, electromagnetism.
  \item Fourier series, boundary problems, and differential equations take center stage.
  \item Physical meaning guides the construction of formal tools.
\end{itemize}

\bigskip

\textbf{8. French Continuity and Transcendence} \hfill \textit{(Hermite, Poincaré)}

\begin{itemize}
  \item Hermite proves transcendence (e.g. $e$).
  \item Poincaré explores differential equations and chaos.
  \item Analysis moves toward dynamical systems and qualitative behavior.
\end{itemize}

\newpage

\section*{IV. Iterative Approaches to a Core Problem}

A central question in analysis:

\begin{quote}
\textbf{What does it mean for a function to be continuous?}
\end{quote}

\noindent
Each school of analysis answered this differently — and in doing so, revealed new possibilities, new dangers, and new definitions of mathematical truth.

\bigskip

\textbf{1. Pre-Rigorous Intuition} \hfill \textit{(Euler)}

\begin{itemize}
  \item A function is continuous if it can be drawn without lifting the pen.
  \item Continuity is a geometric notion — obvious when seen.
  \item Discontinuities are rare, visible, and usually removable.
\end{itemize}

\bigskip

\textbf{2. Cauchy’s Reformulation}

\begin{itemize}
  \item A function is continuous if the limit exists and equals the value at the point.
  \item Formulated using sequences: continuity as limit-preserving.
  \item Still grounded in geometric intuition — but formalized.
\end{itemize}

\bigskip

\textbf{3. Weierstrass’s $\varepsilon$–$\delta$ Definition}

\begin{itemize}
  \item Continuity is a local condition defined without reference to motion or graph.
  \item For every $\varepsilon > 0$, there exists a $\delta > 0$…
  \item This framework admits strange functions: continuous, nowhere differentiable.
\end{itemize}

\bigskip

\textbf{4. Riemann’s Perspective}

\begin{itemize}
  \item Continuity is examined in the context of integration and complex analysis.
  \item Some discontinuous functions are still integrable — the nature of discontinuity matters.
  \item Continuity becomes one aspect of analytic structure.
\end{itemize}

\bigskip

\textbf{5. Cantor and the Set-Theoretic View}

\begin{itemize}
  \item Continuity is about the behavior of preimages of open sets.
  \item Introduces ideas of pointwise vs. uniform convergence.
  \item Distinguishes between countable and uncountable sets of discontinuities.
\end{itemize}

\bigskip

\textbf{6. Poincaré and Beyond}

\begin{itemize}
  \item Continuity is no longer a guarantee of predictability.
  \item Qualitative behavior — not just limit behavior — becomes central.
  \item Dynamical systems expose sensitivity within continuous flows.
\end{itemize}

\bigskip

\textit{From visualization to limit.  
From limit to formal logic.  
From logic to abstraction.  
From abstraction to chaos.}

Continuity survived each shift — but its meaning never stayed fixed.

\newpage

\section*{V. Closing Dialectic}

\noindent
Analysis is where mathematics learned to doubt itself —  
and from doubt, to rebuild.\\

\noindent
It asked questions no longer about values, but about definitions:
\begin{quote}
What is a limit?  
What is a function?  
What is a number?  
What is infinity?
\end{quote}

\vspace{1em}

\noindent
It took the intuitive tools of calculus —  
motion, tangent, area — and subjected them to discipline.

\begin{itemize}
  \item Where Euler calculated, Weierstrass proved.
  \item Where Newton saw fluxions, Cauchy demanded limits.
  \item Where geometry once guided, set theory now governed.
\end{itemize}

\noindent
The real line became a construction.  
Continuity became a condition.  
Infinity became a landscape with coordinates.\\

\noindent
But with this rigor came abstraction:
- Functions no longer had graphs.
- Convergence no longer implied insight.
- Continuity no longer guaranteed comprehension.

Analysis did not collapse under this weight.  
It ascended.\\

\noindent
And in that ascent, it gave modern mathematics its tone:  
\begin{quote}
No result without rigor.  
No structure without definition.  
No truth without foundation.
\end{quote}

\noindent
This was analysis —  
not the end of intuition,  
but the beginning of a much greater degree of certainty.


\end{document}