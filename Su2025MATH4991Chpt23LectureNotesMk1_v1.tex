\documentclass[9pt]{article}
\usepackage{amsmath, amssymb, geometry, graphicx}
\usepackage{titlesec}
\titleformat{\section}[block]{\large\bfseries}{\thesection}{1em}{}
\titleformat{\subsection}[runin]{\bfseries}{}{0pt}{}[.]

\begin{document}

\begin{center}
\Large\textbf{Chapter 23 – Twentieth-Century Legacies} \\
\large Harley Caham Combest \\
\large Su2025 MATH4991 Lecture Notes – Mk1
\end{center}

\vspace{1em}

\subsection*{I. Cultural Invocation}

\begin{itemize}
  \item \textbf{Civilization}: Global Mathematical Culture in the Long Twentieth Century
  \item \textbf{Time Period}: c. 1890--1990
  \item \textbf{Epochs}: Fin de Siècle Foundations, Wars and Migrations, Cold War Ascendancy, Global Modernism
  \item \textbf{Figures}: Henri Poincaré, David Hilbert, Bertrand Russell, Emmy Noether, L. E. J. Brouwer, Felix Hausdorff, John von Neumann, Alan Turing, Nicolas Bourbaki, Alexander Grothendieck
\end{itemize}

\noindent
The twentieth century did not inherit mathematics—it restructured it.

\medskip

\noindent
Institutions expanded. Notations evolved. Abstraction crystallized. From the ashes of war and the ferment of foundations came a mathematics no longer provincial or partitioned—but global, formal, and conceptually interlinked.

\medskip

\noindent
A new kind of power emerged: the ability not merely to solve problems, but to define the very framework in which problems could be understood.

\medskip

\noindent
Poincaré bridged physics and philosophy. Hilbert recast geometry, logic, and number theory into axioms and problems. Lebesgue and Borel redefined integration through measure. Noether turned algebra into language. Bourbaki rebuilt the edifice in structural glass. And from logic's edge, Gödel declared what could not be proven, even in paradise.

\medskip

\noindent
War, migration, and exile scattered minds—and seeded new fields. \\
Topology became algebra. Algebra became category. Computation became destiny. The pure collided with the practical—Turing machines, atomic structures, cybernetic loops.

\medskip

\noindent
This was not merely mathematics as it had been. \\
It was mathematics as architecture, recursion, and resistance.

\newpage

\subsection*{II. Faces of the Era}

\paragraph{Henri Poincaré (1854--1912)}
\begin{itemize}
  \item Founder of \textit{algebraic topology} via \textit{analysis situs}; coined Betti numbers and pioneered fundamental groups.
  \item Developed \textit{automorphic functions}, zeta-Fuchsian systems, and qualitative analysis of differential equations.
  \item Authored vast work in \textit{celestial mechanics}, chaos theory, and mathematical physics.
  \item Bridged science and philosophy with popular writings and a universalist mindset rivaling Gauss.
\end{itemize}

\paragraph{David Hilbert (1862--1943)}
\begin{itemize}
  \item Transformed the foundation of mathematics through \textit{Hilbert’s Problems} (1900) and \textit{axiomatic geometry}.
  \item Unified algebra, number theory, and invariant theory with abstraction and logical formalism.
  \item Advanced the study of \textit{integral equations} and \textit{Hilbert spaces}, foundational to functional analysis.
  \item Engaged with mathematical physics and \textit{axiomatized relativity and quantum theory}.
\end{itemize}

\paragraph{Emmy Noether (1882--1935)}
\begin{itemize}
  \item Revolutionized \textit{abstract algebra} via ideals, chain conditions, and structural methods.
  \item \textit{Noether’s Theorem} linked symmetry and conservation laws, core to modern physics.
  \item Trained a generation of algebraists, influencing ring theory, representation theory, and topology.
\end{itemize}

\paragraph{Henri Lebesgue (1875--1941)}
\begin{itemize}
  \item Reconstructed \textit{integration theory} using measure theory, extending Riemann’s domain.
  \item Developed the \textit{Lebesgue integral}, key to probability, real analysis, and functional spaces.
  \item Feared over-abstraction, yet his work anchored rigor across the discipline.
\end{itemize}

\paragraph{Felix Hausdorff (1868--1942)}
\begin{itemize}
  \item Founded \textit{point-set topology}, formalizing continuity, compactness, and separation axioms.
  \item Introduced the modern concept of \textit{topological spaces} and the four Hausdorff axioms.
  \item His \textit{Grundzüge der Mengenlehre} (1914) remains foundational to modern topology.
\end{itemize}

\paragraph{John von Neumann (1903--1957)}
\begin{itemize}
  \item Architect of \textit{game theory}, quantum logic, and the \textit{modern computer architecture}.
  \item Defined and axiomatized \textit{Hilbert spaces}, vital for quantum mechanics and functional analysis.
  \item Contributed to logic, set theory, economics, and the foundations of computing.
\end{itemize}

\paragraph{Alan Turing (1913--1954)}
\begin{itemize}
  \item Originated the concept of the \textit{Turing machine} and computability theory.
  \item Proved the \textit{undecidability} of the Entscheidungsproblem.
  \item Played a decisive role in cryptanalysis during WWII; foundational figure in artificial intelligence.
\end{itemize}

\paragraph{Nicolas Bourbaki (pseudonymous group, fl. 1930s--1980s)}
\begin{itemize}
  \item Rewrote mathematics from the ground up using \textit{axiomatic and structural formalism}.
  \item Produced \textit{Éléments de mathématique}, a vast treatise emphasizing rigor and abstraction.
  \item Unified algebra, topology, and analysis under \textit{set-theoretic frameworks} and \textit{category theory}.
\end{itemize}

\paragraph{Kurt Gödel (1906--1978)}
\begin{itemize}
  \item Proved the \textit{incompleteness theorems}, showing that no consistent system of arithmetic can be both complete and provably consistent.
  \item Shattered Hilbert’s formalist program; initiated \textit{metamathematics} and undecidability as formal constructs.
  \item His results form the boundary of what mathematics can know about itself.
\end{itemize}

\newpage

\subsection*{III. Works of the Era}

\paragraph{Henri Poincaré}
\begin{itemize}
  \item \textit{Analysis Situs} (1895): \\
  \quad– Foundation of algebraic topology; introduced Betti numbers and fundamental groups.
  \item \textit{Méthodes Nouvelles de la Mécanique Céleste} (1892--1899): \\
  \quad– Advanced qualitative methods in differential equations and celestial mechanics.
  \item \textit{On the Properties of Functions Defined by PDEs} (Doctoral thesis, 1879): \\
  \quad– Early framing of automorphic functions and singularity classification.
  \item \textit{Last Memoir on Diophantine Equations} (1901): \\
  \quad– Pioneering approach to rational points on algebraic curves; influenced Mordell and Weil.
\end{itemize}

\paragraph{David Hilbert}
\begin{itemize}
  \item \textit{Grundlagen der Geometrie} (1899): \\
  \quad– Axiomatic reformulation of Euclidean geometry; established formal structural rigor.
  \item \textit{Zahlbericht} (1897): \\
  \quad– Unified and extended algebraic number theory; emphasized abstraction and systemization.
  \item \textit{Hilbert’s Problems} (1900): \\
  \quad– Twenty-three problems presented at the Paris ICM; shaped the twentieth century’s research agenda.
  \item \textit{Papers on Integral Equations} (1904--1910): \\
  \quad– Founded functional analysis and theory of Hilbert spaces.
  \item \textit{Foundations of Arithmetic and Logic} (with Bernays and Ackermann): \\
  \quad– Attempted to secure consistency in mathematics; later challenged by Gödel.
\end{itemize}

\paragraph{Emmy Noether}
\begin{itemize}
  \item \textit{Ideal Theory in Rings and Fields} (1921): \\
  \quad– Introduced ascending/descending chain conditions; algebraic abstraction for structure.
  \item \textit{Noether’s Theorem} (1918): \\
  \quad– Linked symmetries in physics to conservation laws; cornerstone in theoretical physics.
  \item \textit{Contributions to Representation Theory and Ring Theory}: \\
  \quad– Created foundational language for modern abstract algebra.
\end{itemize}

\paragraph{Henri Lebesgue}
\begin{itemize}
  \item \textit{Leçons sur l’Intégration} (1904): \\
  \quad– Defined the Lebesgue integral using measure theory.
  \item \textit{Leçons sur les Séries Trigonométriques} (1903): \\
  \quad– Applied measure theory to function spaces and Fourier series.
\end{itemize}

\paragraph{Felix Hausdorff}
\begin{itemize}
  \item \textit{Grundzüge der Mengenlehre} (1914): \\
  \quad– Defined topological spaces, neighborhood systems, and separation axioms (T1--T4).
  \item \textit{Formalization of Continuity and Compactness}: \\
  \quad– Made topology a rigorous, autonomous discipline.
\end{itemize}

\paragraph{Nicolas Bourbaki}
\begin{itemize}
  \item \textit{Éléments de Mathématique} (1939--1980s): \\
  \quad– Axiomatic reformation of modern mathematics; structured by set theory and category. \\
  \quad– Volumes: Set Theory, Algebra, General Topology, Integration, Lie Groups, Spectral Theory. \\
  \quad– Unified language across disciplines and trained generations of mathematicians.
\end{itemize}

\paragraph{Kurt Gödel}
\begin{itemize}
  \item \textit{On Formally Undecidable Propositions\dots} (1931): \\
  \quad– Incompleteness Theorems: no consistent formal system of arithmetic can prove its own consistency.
  \item \textit{Consistency of the Axiom of Choice with ZF Set Theory} (1940): \\
  \quad– Major advance in logic and foundations; precursor to independence results.
\end{itemize}

\paragraph{Alan Turing}
\begin{itemize}
  \item \textit{On Computable Numbers\dots} (1937): \\
  \quad– Defined the Turing machine; solved the Entscheidungsproblem negatively.
  \item \textit{WWII Codebreaking Work at Bletchley Park}: \\
  \quad– Practical computing applications; laid groundwork for digital computation and AI.
\end{itemize}

\newpage


\subsubsection*{Concept Note: Betti Numbers, Automorphic Forms, and the Entscheidungsproblem}


\paragraph{Betti Numbers}
\textit{Betti numbers} are topological invariants that count the number of independent $k$-dimensional ``holes'' in a topological space. Named after the Italian mathematician \textbf{Enrico Betti}, they arise in algebraic topology via \textit{homology groups}.

\medskip

\noindent Formally, the $k^\text{th}$ Betti number $b_k$ is defined as:
\[
b_k = \text{rank of the } k^\text{th} \text{ homology group } H_k(X)
\]
where $X$ is a topological space.

\begin{itemize}
  \item $b_0$: number of connected components
  \item $b_1$: number of 1-dimensional loops or holes (e.g., handles)
  \item $b_2$: number of enclosed voids (e.g., cavities in 3D)
\end{itemize}

\noindent For example, a torus has:
\[
b_0 = 1, \quad b_1 = 2, \quad b_2 = 1
\]

\noindent Betti numbers are essential in classifying topological spaces up to homotopy equivalence and are fundamental in algebraic topology, especially in the study of manifolds.

\bigskip

\paragraph{Automorphic Forms}
An \textit{automorphic form} is a complex-valued function defined on a mathematical space (often the upper half-plane) that remains invariant (or transforms in a prescribed way) under the action of a discrete group of transformations—typically Möbius (linear fractional) transformations:
\[
z \mapsto \frac{az + b}{cz + d}, \quad \text{where } a,b,c,d \in \mathbb{R} \text{ (or }\mathbb{Z}\text{)},\quad ad - bc \neq 0
\]

These generalize:
\begin{itemize}
  \item \textit{Trigonometric functions} (periodic under translations)
  \item \textit{Elliptic functions} (doubly periodic)
  \item \textit{Modular forms} (invariant under the modular group)
\end{itemize}

Automorphic forms are central to number theory, representation theory, and the Langlands program. Poincaré pioneered the theory by classifying automorphic functions and connecting them to differential equations.

\bigskip

\paragraph{Entscheidungsproblem (``Decision Problem'')}
Posed by David Hilbert in the early twentieth century, the \textit{Entscheidungsproblem} asks:

\begin{quote}
\textit{Is there a finite, mechanical procedure that can determine whether any given mathematical statement is provable from a given set of axioms?}
\end{quote}

In modern terms, it seeks a universal algorithm to decide the truth or provability of statements in first-order logic.

\medskip

\noindent \textbf{Alan Turing (1936)} and \textbf{Alonzo Church (1935)} independently showed that the answer is \textbf{no}---there is no such algorithm. Turing formalized this using the concept of the \textit{Turing machine}, and demonstrated that certain problems (now called \textit{undecidable problems}) lie beyond the reach of algorithmic computation.

\medskip

This result became a foundational limit in mathematical logic and computer science---closely linked to \textit{Gödel’s incompleteness theorems}.

\newpage

\subsection*{IV. Historical Overview}

\paragraph{
The twentieth century marks the moment mathematics outgrew its own skin.}

Not merely an accumulation of results, this era witnessed the \textit{reconstitution} of the field: its axioms reformulated, its boundaries redrawn, and its functions embedded into science, logic, computation, and culture.

\begin{enumerate}
  \item \textbf{From Unification to Abstraction} \\
  -- Poincaré’s integration of mechanics, functions, and topology signaled a new universality. \\
  -- Hilbert sought to unify mathematics through formal axiomatics and challenge. \\
  -- Noether, Hausdorff, and Lebesgue pushed abstraction to its structural edge: rings, topologies, measures.

  \item \textbf{Formalism Confronts Its Limits} \\
  -- Hilbert’s dream of securing mathematics through logic faltered in the face of Gödel’s incompleteness. \\
  -- Turing’s response to the \textit{Entscheidungsproblem} placed irrevocable boundaries on algorithmic reasoning. \\
  -- Mathematics discovered that its own foundations were necessarily incomplete, undecidable, or unprovable within.

  \item \textbf{Topology Becomes Language} \\
  -- What began as ``rubber-sheet geometry'' became the connective tissue of all fields. \\
  -- From Poincaré’s Betti numbers to Hausdorff’s axioms to Fréchet spaces, topology redefined continuity and convergence. \\
  -- Category theory and homological algebra abstracted relationships across geometry, algebra, and analysis.

  \item \textbf{Algebra Ascends} \\
  -- Noetherian structures, ideal theory, and representation theory turned algebra into a universal syntax. \\
  -- The field-theoretic legacy of Dedekind, Steinitz, and Artin converged with geometry in Grothendieck's schemes.

  \item \textbf{Probability and Measure Merge} \\
  -- Measure theory (Lebesgue, Borel) gave rigor to the statistical sciences. \\
  -- Kolmogorov’s axioms and Markov chains framed modern stochastic processes. \\
  -- Probability lost its philosophical ambiguity and gained mathematical precision.

  \item \textbf{Computation Rewrites the Possible} \\
  -- Turing formalized computability. von Neumann built the architecture. \\
  -- Cryptography, algorithms, and logic became formal disciplines. \\
  -- Computers did not merely solve mathematics—they became part of its definition.

  \item \textbf{War and Exile Shape the Field} \\
  -- The rise of fascism shattered German and Eastern European mathematics. \\
  -- Emigres like Noether, Weyl, Artin, and Gödel carried brilliance to the US and beyond. \\
  -- New mathematical centers formed from rupture: Princeton, Chicago, Moscow, Paris, Berkeley.

  \item \textbf{Bourbaki and the Structural Age} \\
  -- Under the mask of one man, a collective rewrote the foundations of the entire field. \\
  -- Set theory, abstraction, and axiomatic rigor became the scaffolding of 20th-century pedagogy.

  \item \textbf{The Fracturing and Fusion of Fields} \\
  -- Algebra, topology, geometry, and analysis no longer operated in silos. \\
  -- Homological methods, category theory, and sheaf cohomology fused languages. \\
  -- Mathematics became self-aware of its own unity.
\end{enumerate}

\newpage

\subsection*{V. Problem--Solution Cycle}

\paragraph{Problem 1: Poincaré’s Genus--Singularity Formula (1880s)}
\textbf{Statement.} Classify the singularities of a differential equation using geometric topology. \\
\textbf{Solution.}
\begin{itemize}
  \item Consider the differential equation \( F(x, y, y') = 0 \) defined on an algebraic surface.
  \item Let \( N \) = number of nodes, \( F \) = number of foci, \( S \) = number of saddles, and \( p \) = genus.
  \item Poincaré’s relation:
  \[
  N + F - S = 2 - 2p
  \]
  \item This generalizes Euler's formula to the realm of differential geometry via topological classification.
\end{itemize}

\paragraph{Problem 2: Gödel’s Incompleteness Theorem (1931)}
\textbf{Statement.} Can arithmetic prove its own consistency? \\
\textbf{Solution.}
\begin{itemize}
  \item Construct a formal system \( S \) capable of encoding arithmetic.
  \item Use Gödel numbering to create a self-referential proposition \( G \): ``\( G \) is not provable in \( S \).''
  \item If \( G \) is provable, then it’s false \( \Rightarrow \) contradiction.
  \item If \( G \) is not provable, then it is true \( \Rightarrow \) incompleteness.
  \item Therefore:
  \[
  \text{No consistent formal system of arithmetic can prove all arithmetic truths.}
  \]
\end{itemize}

\paragraph{Problem 3: Hilbert’s Third Problem (1900)}
\textbf{Statement.} Can two tetrahedra of equal volume and base-height be decomposed into congruent parts? \\
\textbf{Solution.}
\begin{itemize}
  \item Max Dehn (1902) answered: No.
  \item Introduced the \textit{Dehn invariant}, which obstructs dissection congruence in 3D.
  \item Unlike 2D polygons (equal area \( \Rightarrow \) scissor congruent), 3D solids require additional invariants.
  \item Thus, Hilbert’s Third Problem was the first to be resolved---negatively.
\end{itemize}

\paragraph{Problem 4: Turing’s Negative Solution to the \textit{Entscheidungsproblem} (1936)}
\textbf{Statement.} Is there a universal algorithm to decide truth in first-order logic? \\
\textbf{Solution.}
\begin{itemize}
  \item Turing defines the \textit{Turing machine}, a formal model of computation.
  \item Constructs the \textit{halting problem}: no machine can decide whether another machine halts.
  \item Hence, no general algorithm can decide all mathematical truths.
  \item Outcome: undecidability becomes a permanent feature of mathematics.
\end{itemize}

\paragraph{Problem 5: Lebesgue’s Measure Zero Example (1904)}
\textbf{Statement.} Integrate a function that is 0 on rationals, 1 on irrationals in \([0,1]\). \\
\textbf{Solution.}
\begin{itemize}
  \item The function is discontinuous everywhere \( \Rightarrow \) not Riemann integrable.
  \item However, the \textit{set of rationals} has measure zero.
  \item Lebesgue integral:
  \[
  \int_0^1 f(x)\, dx = 1 \cdot 1 + 0 \cdot 0 = 1
  \]
  \item Established Lebesgue’s theory as a generalization of Riemann’s.
\end{itemize}

\paragraph{Problem 6: Weil Conjectures and Deligne’s Proof (1974)}
\textbf{Statement.} Do zeta functions over finite fields satisfy a Riemann-like hypothesis? \\
\textbf{Solution.}
\begin{itemize}
  \item Weil conjectured deep analogies between topology and number theory.
  \item Grothendieck developed \textit{étale cohomology} to approach it.
  \item Pierre Deligne proved the final (Riemann) conjecture in 1974.
  \item Unified algebraic geometry and arithmetic under cohomological tools.
\end{itemize}

\newpage

\subsection*{VI. Closing Dialectic}

\paragraph{
The twentieth century did not give us closure. It gave us \textit{recursion}.
}

Poincaré framed mathematics as intuition guided by rigor. Hilbert demanded rigor become its own intuition. Noether turned structure into language. Gödel proved that language cannot contain itself.\\

This century was not about finding answers---it was about discovering the boundaries of what answers are.\\

\begin{itemize}
  \item \textbf{Topology} ceased being shape and became structure.
  \item \textbf{Algebra} ceased being computation and became syntax.
  \item \textbf{Logic} ceased being certainty and became limit.
  \item \textbf{Computation} ceased being tool and became definition.
\end{itemize}

Mathematics fractured---then fused.\\

\noindent
What had once been separate domains---geometry, algebra, analysis, logic, physics---began to speak to each other in a common tongue: \textit{abstraction}. Theorems became categories. Equations became functors. Proof became architecture.\\

\noindent
But this unification came with a paradox: the more mathematics generalized, the more it revealed the impossibility of completing itself.

\bigskip

Hilbert dreamed of a system that could capture all truths. Gödel proved no such system could exist. Turing showed there is no universal machine of judgment. Yet mathematics endured---not by resolving these limits, but by incorporating them into its nature.

\bigskip

And so, mathematics in the twentieth century became not merely a discipline, but a mirror:

\begin{quote}
To study its truths was to confront our finitude. \\
To extend its form was to trace the architecture of thought itself.
\end{quote}

No longer the language of certainty, mathematics became the discipline of \textit{disciplined uncertainty} in a more unified form.



\end{document}