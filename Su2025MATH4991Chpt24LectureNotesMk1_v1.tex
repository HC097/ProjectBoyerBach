\documentclass[9pt]{article}
\usepackage{amsmath, amssymb, geometry, graphicx}
\usepackage{titlesec}
\titleformat{\section}[block]{\large\bfseries}{\thesection}{1em}{}
\titleformat{\subsection}[runin]{\bfseries}{}{0pt}{}[.]

\begin{document}

\begin{center}
\Large\textbf{Chapter 24 – Recent Trends} \\
\large Harley Caham Combest \\
\large Su2025 MATH4991 Lecture Notes – Mk1
\end{center}

\vspace{1em}

\subsection*{I. Cultural Invocation}

\begin{itemize}
  \item \textbf{Civilization}: Globalized Mathematical Modernity
  \item \textbf{Time Period}: c.\ 1970--Present
  \item \textbf{Epochs}: Post-Hilbertian Collapse, Computational Ascendancy, Global Diffusion
  \item \textbf{Figures}: Kenneth Appel, Wolfgang Haken, Daniel Gorenstein, Andrew Wiles, Grigori Perelman, Georges Gonthier, Shing-Tung Yau
\end{itemize}

The final decades of the twentieth century were not a climax. They were a rupture.

Mathematics became a battlefield of paradox: proofs invisible to humans; groups classified by collective will; a childhood riddle undone by elliptic curves. Geometry, once dethroned, returned---not with compass and straightedge, but with Ricci flow and software verifiers.

What had once required chalk now demanded machines. What had once required insight now required infrastructure.

Yet at the same time, the flame endured. Problems declared too vast to finish were finished. Conjectures mocked for their absurd longevity---Fermat’s, Poincaré’s---fell, not by solitary leaps, but by centuries of accumulated arc.

\medskip

This was the age when:
\begin{itemize}
  \item Proof and computation crossed paths---and questioned each other.
  \item Mathematical legitimacy no longer depended solely on elegance, but on code, consensus, and complexity.
  \item Nations once peripheral became centers of excellence.
  \item Formalism met reality---not as theory, but as architecture.
\end{itemize}

The question was no longer ``Can this be solved?'' \\
It was: ``Can we agree what solving even means?''

\medskip

This was not the era of axioms. It was the era of \textit{verification}. \\
Not the age of founding principles, but of \textit{closing epics}.

\medskip

And yet---beneath it all---doubt remained, disciplined. \\
Mathematics continued not because all was knowable, but because rigor remained sacred.

\newpage

\subsection*{II. Faces of the Era}

\subsubsection*{Concept Note: The Four Color Theorem and Poincaré's Conjecture}

\paragraph{Four Color Theorem}
\textit{Statement.} Any map drawn on a flat surface (or a sphere), divided into contiguous regions, can be colored using no more than four colors so that no two regions sharing a boundary segment receive the same color.

\medskip

\noindent More precisely: if each region of the map is represented by a ``node,'' and shared borders are represented by ``connections,'' then no more than four colors are ever needed to color all nodes so that connected nodes differ in color.

\medskip

\noindent First proposed by Francis Guthrie in 1852, the problem defied proof for over a century. It was finally resolved in 1976 by \textbf{Kenneth Appel} and \textbf{Wolfgang Haken}, who used a computer to check thousands of possible cases. This was the first major theorem to rely on computer assistance, prompting philosophical debate about what constitutes a ``proof.''

\medskip

\paragraph{Poincaré's Conjecture}
\textit{Statement.} Every closed, simply connected three-dimensional space without boundary is topologically equivalent to a three-dimensional sphere:
\[
\text{If a 3-manifold has no holes and no edges, then it is homeomorphic to } S^3
\]

\noindent Formulated by Henri Poincaré in 1904, the conjecture resisted proof for nearly a century. In the early 2000s, \textbf{Grigori Perelman} resolved it by applying \textit{Ricci flow with surgery}, a geometric evolution equation. His work also confirmed the broader \textit{Geometrization Conjecture} of William Thurston.

\medskip

\noindent Perelman’s proof was revolutionary not only in method but in delivery: he bypassed journals, refused awards, and questioned the culture of mathematical prestige. The Poincaré conjecture thus closed not only a mathematical epoch—but opened a new ethical one.


\paragraph{Kenneth Appel (1932--2013) \& Wolfgang Haken (b.\ 1928)}
\begin{itemize}
  \item Co-authors of the first \textit{computer-assisted proof} of the Four-Color Theorem (1976).
  \item Introduced the method of \textit{reducibility and unavoidability} validated via 1,200 hours of computation.
  \item Sparked philosophical debate on the nature of proof and machine verification.
\end{itemize}

\paragraph{Daniel Gorenstein (1923--1992)}
\begin{itemize}
  \item Architect and coordinator of the \textit{Classification of Finite Simple Groups (CFSG)}.
  \item Unified decades of disparate results into a massive, multi-author collaborative framework.
  \item Brought rigor, editorial control, and structure to one of the most ambitious projects in mathematical history.
\end{itemize}

\paragraph{Andrew Wiles (b.\ 1953)}
\begin{itemize}
  \item Proved \textit{Fermat’s Last Theorem} in 1994 by resolving the modularity of semistable elliptic curves.
  \item Synthesized the ideas of \textit{modular forms, elliptic curves, and Galois representations}.
  \item Closed a 350-year arc of mathematical longing with a modern arsenal of algebraic geometry.
\end{itemize}

\paragraph{Grigori Perelman (b.\ 1966)}
\begin{itemize}
  \item Solved the \textit{Poincaré Conjecture} and proved Thurston’s \textit{Geometrization Conjecture} using Ricci flow and entropy techniques.
  \item Posted three minimalist arXiv papers (2002--2003), bypassing journals and peer review.
  \item Declined the \textit{Fields Medal} (2006) and \textit{Clay Millennium Prize} (2010), citing moral and communal disillusionment.
  \item His proof ignited a \textit{credit controversy}, particularly regarding Huai-Dong Cao and Xi-Ping Zhu’s follow-up work.
  \item \textbf{Shing-Tung Yau} publicly framed the solution in nationalistic terms, drawing criticism and attention in a 2006 \textit{New Yorker} exposé.
  \item Perelman's withdrawal became a symbol of both \textit{mathematical integrity} and principled estrangement from institutional prestige.
\end{itemize}

\paragraph{Georges Gonthier (b.\ 1965)}
\begin{itemize}
  \item Led the formal verification of the \textit{Four-Color Theorem} using the \textit{Coq proof assistant} (2005).
  \item Reframed mathematical proof as a \textit{programming discipline}, introducing industrial-level reliability to verification.
  \item Opened the door to \textit{machine-checked mathematics} as a viable epistemology.
\end{itemize}

\paragraph{Shing-Tung Yau (b.\ 1949)}
\begin{itemize}
  \item Fields Medalist and co-developer of \textit{Calabi--Yau manifolds}, essential to string theory and geometry.
  \item Prominent contributor to \textit{differential geometry}, \textit{Ricci curvature}, and \textit{geometric analysis}.
  \item Influential public figure in global mathematics; promoted Chinese mathematical achievement and national presence.
\end{itemize}

\paragraph{Robin Thomas, Paul Seymour, Neil Robertson, Daniel Sanders}
\begin{itemize}
  \item Provided a \textit{simplified and modular computer-assisted proof} of the Four-Color Theorem (1997).
  \item Reduced configuration cases to 633 and streamlined discharging techniques.
  \item Affirmed the \textit{feasibility of collaborative, computer-verified large-scale proofs}.
\end{itemize}

\newpage

\subsection*{III. Works of the Era}

\paragraph{Appel \& Haken – \textit{Every Planar Map is Four Colorable} (1976)}
\begin{itemize}
  \item First major \textit{computer-assisted proof} in mathematics.
  \item Introduced a method to reduce all cases to a finite set of \textit{1,936 configurations}, then showed no counterexample could contain any of them.
  \item Published in a series of papers and later expanded in their 1989 book-length technical guide.
  \item Catalyzed debate on the nature of proof and the legitimacy of uncheckable computation.
\end{itemize}

\paragraph{Robertson, Sanders, Seymour, Thomas – \textit{Simplified Four-Color Proof} (1997)}
\begin{itemize}
  \item Reduced the number of configurations to \textit{633} and discharging rules to \textit{32}.
  \item Improved algorithmic transparency and mitigated criticism of the original Appel--Haken approach.
  \item Cited concerns about compiler trust and hardware fallibility---raising early questions in \textit{software verification ethics}.
\end{itemize}

\paragraph{Gonthier \& Werner – \textit{Formal Proof in Coq} (2005)}
\begin{itemize}
  \item Verified the four-color theorem using the \textit{Coq proof assistant}.
  \item Provided not just a proof, but a \textit{machine-verified certificate of correctness}.
  \item Marked a shift from proof as rhetoric to proof as \textit{code artifact}.
\end{itemize}

\paragraph{Gorenstein et al. – \textit{Classification of Finite Simple Groups} (1983--2004)}
\begin{itemize}
  \item Unified prior work into a \textit{“Periodic Table” of finite simple groups}.
  \item Final proof spanned \textit{over 10,000 pages}, culminating in the Aschbacher--Smith volumes (2004).
  \item Noted for its size, collaborative nature, and relative lack of external applications---provoking reflection on \textit{intrinsic value in mathematics}.
\end{itemize}

\paragraph{Wiles – \textit{Modular Elliptic Curves and Fermat’s Last Theorem} (1995)}
\begin{itemize}
  \item Proved the \textit{semistable case of the Taniyama--Shimura--Weil conjecture}, which implied Fermat’s Last Theorem.
  \item Employed tools from \textit{Iwasawa theory, Hecke algebras, Galois representations}, and \textit{modular forms}.
  \item Joint supplement with Richard Taylor resolved the critical gap in the original argument.
\end{itemize}

\paragraph{Perelman – \textit{Ricci Flow and the Geometrization of 3-Manifolds} (2002--2003)}
\begin{itemize}
  \item Three papers posted on the arXiv outlined a proof of the \textit{Geometrization Conjecture}, implying the \textit{Poincaré Conjecture}.
  \item Introduced \textit{entropy formulas}, \textit{surgery techniques}, and refined \textit{Hamilton’s Ricci flow}.
  \item Verified independently by several groups; Perelman refused publication, prizes, and interviews.
  \item Changed the discourse on authorship, proof sufficiency, and mathematical ethics.
\end{itemize}

\newpage

\subsection*{IV. Historical Overview}

\paragraph{The late twentieth and early twenty-first centuries witnessed a decisive transformation in the identity, scale, and method of mathematics.}

\vspace{1em}

What had once been a solo endeavor of chalk and blackboard became a networked enterprise---hybridized with computation, institutionalized into collaborations, and globalized across continents.

\begin{enumerate}
  \item \textbf{The Rise of Machine-Verified Proofs} \\
  -- Appel and Haken’s proof of the Four-Color Theorem (1976) marked a historic threshold: the first theorem whose truth exceeded human comprehension in raw detail. \\
  -- The response was mixed---celebration, discomfort, and the birth of ``proof ethics.'' \\
  -- Later work by Gonthier, Robertson, and others normalized machine assistance in rigorous verification.

  \item \textbf{Completion of Monumental Projects} \\
  -- Gorenstein and collaborators completed the \textit{Classification of Finite Simple Groups}---a project spanning decades and involving hundreds of mathematicians. \\
  -- Its 10{,}000-page sprawl challenged the community’s capacity to verify, understand, or even read its proof in full. \\
  -- The CFSG posed a deeper question: is a proof still a proof if no single mind can encompass it?

  \item \textbf{Resolution of Ancient Conjectures} \\
  -- \textit{Fermat’s Last Theorem}, proven by Wiles, did not rely on 17th-century arithmetic, but on 20th-century modularity, elliptic curves, and Galois theory. \\
  -- \textit{Poincaré’s Conjecture}, resolved by Perelman, leveraged geometric analysis, Ricci flow, and a refusal to participate in traditional academic rewards.

  \item \textbf{A New Role for Computers} \\
  -- Computers moved from heuristic tool to ontological participant. \\
  -- Proofs became code. Code became argument. The nature of truth shifted toward \textit{verifiability by machine}. \\
  -- This created tension: traditional proof as human persuasion versus code as executable verification.

  \item \textbf{Globalization of Mathematical Culture} \\
  -- Mathematical leadership shifted beyond Western Europe and North America. \\
  -- Contributions emerged from China, Russia, and elsewhere---facilitated by internet-era collaboration and open access. \\
  -- The \textit{arXiv}, launched in 1991, became the central node in decentralized knowledge circulation.

  \item \textbf{Erosion of Romantic Individualism} \\
  -- While figures like Wiles and Perelman embodied the lone genius myth, their work relied on centuries of scaffolding. \\
  -- Massive projects and formal verification diluted the myth of solitary insight. \\
  -- Mathematics became less about heroic leaps---and more about distributed rigor.

  \item \textbf{A New Philosophical Uncertainty} \\
  -- The rise of computer proofs, modular structures, and high-dimensional tools prompted new debates: \\
  \quad $\circ$ What counts as a ``proof''? \\
  \quad $\circ$ Who (or what) gets credit? \\
  \quad $\circ$ Can trust and truth be reconciled in a machine age?
\end{enumerate}

\newpage

\subsection*{V. Problem--Solution Cycle}

\paragraph{Problem 1: Four-Color Conundrum (1852--1976)}
\textbf{Statement.} Can every map be colored with no more than four colors so that no two adjacent regions share the same color? \\
\textbf{Solution.}
\begin{itemize}
  \item Francis Guthrie first posed the problem in 1852.
  \item Appel and Haken (1976) decomposed all possible maps into \textit{1,936 reducible configurations}.
  \item Using a \textit{computer program}, they verified each case via brute-force logic.
  \item Philosophical impact: challenged the status of proofs not comprehensible by a single human mind.
\end{itemize}

\paragraph{Problem 2: Verifying the Verification (2005)}
\textbf{Statement.} Can a computer-verified proof of the Four-Color Theorem be trusted independently of compiler assumptions and machine errors? \\
\textbf{Solution.}
\begin{itemize}
  \item Georges Gonthier recoded the entire proof in the \textit{Coq proof assistant}.
  \item Result: a formalized, logic-level derivation certified by foundational axioms.
  \item Introduced \textit{machine-checkable proofs} as a new epistemological standard.
  \item Signaled the dawn of \textit{proof as software artifact}.
\end{itemize}

\paragraph{Problem 3: Completing the Classification (1980s--2004)}
\textbf{Statement.} Can all finite simple groups be listed and classified? \\
\textbf{Solution.}
\begin{itemize}
  \item The classification required identifying all \textit{non-decomposable group types}.
  \item Result: a taxonomy of \textit{26 sporadic groups}, \textit{cyclic}, \textit{alternating}, and \textit{Lie-type groups}.
  \item Final verification spanned 10{,}000+ pages, published across dozens of journals.
  \item A proof of \textit{industrial scale}, prompting questions about collective authorship and comprehension.
\end{itemize}

\paragraph{Problem 4: Fermat’s Last Theorem (1637--1994)}
\textbf{Statement.} Are there whole number solutions to \( x^n + y^n = z^n \) for \( n > 2 \)? \\
\textbf{Solution.}
\begin{itemize}
  \item Andrew Wiles proved that all \textit{semistable elliptic curves} are modular.
  \item Combined deep tools: \textit{Galois representations}, \textit{modular forms}, and \textit{Hecke algebras}.
  \item Linked 17th-century arithmetic to 20th-century algebraic geometry.
  \item The original ``last theorem'' became a corollary to a modern conjecture.
\end{itemize}

\paragraph{Problem 5: Poincaré’s 3-Manifold Puzzle (1904--2003)}
\textbf{Statement.} Is every closed, simply connected 3-manifold topologically a 3-sphere? \\
\textbf{Solution.}
\begin{itemize}
  \item Grigori Perelman applied \textit{Hamilton’s Ricci flow} with surgery to collapse curvature singularities.
  \item Introduced \textit{entropy formulas} and monotonicity methods to guide the flow.
  \item His minimalist arXiv preprints bypassed peer review but withstood full verification.
  \item Result: affirmation of the \textit{Geometrization Conjecture} and resolution of Poincaré’s century-old question.
\end{itemize}

\newpage

\subsection*{VI. Closing Dialectic}

\paragraph{This was not a century of discovery. \\
It was a century of \textit{convergence}.}

\medskip

Proof and machine became indistinguishable. \\

\noindent
Rigor no longer lived on chalkboards---it pulsed through servers. \\

\noindent
Mathematics shed its romantic loneliness and clothed itself in code, collaboration, and controversy.

\medskip

The Four-Color Theorem no longer belonged to vision---it belonged to computation. \\

\noindent
Fermat’s Last Theorem no longer belonged to arithmetic---it belonged to modularity. \\

\noindent
The Poincaré Conjecture no longer belonged to topology---it belonged to flow, entropy, and ascetic refusal.

\medskip

This era forced a reckoning:

\begin{itemize}
  \item What is a proof, if it cannot be read?
  \item What is credit, if it cannot be assigned?
  \item What is truth, if it must be compiled?
\end{itemize}

Mathematics did not answer these questions. \\
It \textit{endured} them.

\medskip

Perelman, Wiles, Gonthier, and Gorenstein each revealed a facet of what the discipline had become: \\
A mirror of the world it once stood apart from. \\
A discipline that no longer claims certainty---but never surrenders rigor.

\medskip

This was not an epilogue to the history of mathematics. \\
It was its \textbf{reforging}---in silicon, in silence, in solidarity.


\end{document}