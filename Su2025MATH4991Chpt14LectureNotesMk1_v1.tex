\documentclass[9pt]{article}
\usepackage{amsmath, amssymb, geometry, graphicx, float, fancyhdr}
\usepackage[most]{tcolorbox}

\geometry{margin=1in}
\pagestyle{fancy}
\fancyhf{}
\rhead{Su2025MATH4991 Lecture Notes}
\lhead{Chapter 14 - Early Modern Problem Solvers}
\cfoot{\thepage}

\title{Chapter 14 -- Early Modern Problem Solvers}
\author{Mr. Harley Caham Combest}
\date{June 2025}

\begin{document}
\maketitle

\section*{I. Cultural Invocation}

\begin{itemize}
    \item \textbf{Civilization:} Early Modern Europe (Flemish Republics, Scottish Highlands, English Universities, Tuscan City-States, Habsburg Domains)
    \item \textbf{Time Period:} c. 1580–1640 AD
    \item \textbf{Figures:} 
    Simon Stevin (decimals, statics), \\
    John Napier (logarithms, rods), \\
    Henry Briggs (common logs), \\
    Jobst Bürgi (independent logarithms), \\
    Galileo Galilei (computing devices, physics), \\
    Edmund Gunter (scales, trigonometry), \\
    William Oughtred (slide rules), \\
    Johannes Kepler (ellipses, volumes)
\end{itemize}


\begin{center}
    \includegraphics[scale=.15]{cover1598.png}
    
    \textit{Map of the territories of Philip II, King of Spain in 1598 (The Iberian Union: 1580 - 1640)}
\end{center}

\vspace{1em}

\begin{quote}
\itshape
In the generation after Viète, Europe shifted again — not in theory, but in tools.\\
Mathematics moved from ink to instrument, from proof to practice.

A new class of thinkers arose — engineers, astronomers, merchants, machinists — who did not debate whether mathematics was Platonic or empirical.\\
They needed answers. And they needed them fast.

So they built rods, scales, wheels, sectors, bones, and engines.\\
They carved logarithms into wood. They revolved parabolas into citrons. They counted in decimals instead of sixtieths.

Galileo measured motion without calculus.\\
Kepler filled wine barrels with indivisibles.\\
Stevin proved gravity before Newton could define it.

These were not philosophers of mathematics.\\
They were \textit{solvers}.

And in their wake came tables, compasses, machines — and the slow collapse of the classical divide between geometry and number.
\end{quote}



\newpage



\section*{II. Faces of the Era}

\textbf{Simon Stevin} (1548–1620)

\begin{itemize}
    \item Flemish engineer and reformer of computation; introduced decimal fractions to a broad public through \textit{De Thiende} (1585).
    \item Applied Archimedean reasoning to statics, pressure, and center of gravity; rediscovered the law of the inclined plane.
    \item Pioneered positional notation for exponents and symbolic representation of roots, inspiring later algebraists and instrument makers.
\end{itemize}

\textbf{John Napier} (1550–1617)

\begin{itemize}
    \item Scottish nobleman and inventor of logarithms; published \textit{Descriptio} (1614) and \textit{Constructio} (1619).
    \item His tables simplified multiplication and revolutionized trigonometric calculation.
    \item Also devised “Napier’s bones” — rods for quick lattice-based multiplication.
\end{itemize}

\textbf{Henry Briggs} (1561–1630)

\begin{itemize}
    \item English mathematician who collaborated with Napier; proposed base-10 (common) logarithms.
    \item Created the \textit{Arithmetica Logarithmica} (1624), tabulating logs to 14 digits — standard for centuries.
    \item Introduced the terms “mantissa” and “characteristic.”
\end{itemize}

\textbf{Jobst Bürgi} (1552–1632)

\begin{itemize}
    \item Swiss clockmaker and astronomer; independently developed logarithmic tables by 1588, published in 1620.
    \item Also constructed geometric instruments and calculating devices rivaling those of Galileo.
    \item Though late to publish, his numerical approach paralleled Napier’s in power and concept.
\end{itemize}

\textbf{Galileo Galilei} (1564–1642)

\begin{itemize}
    \item Italian polymath; created the “geometric and military compass” (1597) — a portable calculating device.
    \item Used simple mechanical means to solve complex physical problems — before formal calculus existed.
    \item His compasses, pulse meters, and inclined-plane experiments exemplified the blend of instrument and insight.
\end{itemize}

\textbf{Edmund Gunter} (1581–1626)

\begin{itemize}
    \item English clergyman and mathematician; devised the “Gunter’s scale” — a logarithmic tool for navigation and land measurement.
    \item Popularized practical trigonometric tables and instruments for surveyors and sailors.
    \item Laid the foundation for the development of the slide rule.
\end{itemize}

\textbf{William Oughtred} (1574–1660)

\begin{itemize}
    \item English mathematician and educator; invented both circular and linear slide rules.
    \item Formalized algebraic notation in his \textit{Clavis Mathematicae}; taught through symbols rather than rhetoric.
    \item Popularized multiplication symbol ($\times$) and advanced compact symbolic reasoning.
\end{itemize}

\textbf{Johannes Kepler} (1571–1630)

\begin{itemize}
    \item German astronomer and geometric thinker; formulated the three laws of planetary motion.
    \item Used indivisibles to calculate areas and volumes — a proto-integral approach decades before Newton and Leibniz.
    \item Applied conic sections, continuity, and spatial reasoning in both astronomy and volumetrics — from ellipses to wine barrels.
\end{itemize}



\newpage


\section*{III. Works of the Era}

\textbf{Simon Stevin — \textit{De Thiende} (1585)}

\begin{itemize}
    \item Introduced decimal fractions systematically for use in everyday arithmetic and engineering.
    \item Designed positional notation for decimal powers (e.g., using circled numbers or superscripts).
    \item Advocated numerical literacy for commercial and civic life — ``to teach all men to calculate without fractions.''
\end{itemize}

\textbf{Simon Stevin — \textit{Statics} (1586)}

\begin{itemize}
    \item Demonstrated the law of the inclined plane using the “wreath of spheres” argument.
    \item Applied geometric exhaustion methods to show centers of gravity and physical equilibria.
    \item Extended Archimedean reasoning to practical applications in hydrostatics and mechanics.
\end{itemize}

\textbf{John Napier — \textit{Descriptio} (1614) and \textit{Constructio} (1619)}

\begin{itemize}
    \item Introduced logarithms as ratios of motion — a geometric and numerical simplification technique.
    \item Defined logarithms via a continuous point moving uniformly and another decelerating geometrically.
    \item Though based on $1/e$, the system anticipated natural logs and enabled dramatic reductions in multiplication effort.
\end{itemize}

\textbf{Henry Briggs — \textit{Arithmetica Logarithmica} (1624)}

\begin{itemize}
    \item Tabulated base-10 (common) logarithms from 1 to 20,000 and 90,000 to 100,000 to 14 decimal places.
    \item Established the logarithm of 1 as 0, and 10 as 1 — standardizing the base-10 system still in use.
    \item Introduced terms “mantissa” (decimal part) and “characteristic” (integer part).
\end{itemize}

\textbf{Jobst Bürgi — \textit{Arithmetische und Geometrische Progress-Tabulen} (1620)}

\begin{itemize}
    \item Independently compiled logarithmic tables using powers of $1 + \tfrac{1}{10^4}$ scaled by $10^8$.
    \item Demonstrated deep understanding of geometric interpolation and exponential growth.
    \item His “red” and “black” number system approximated natural logarithms with remarkable accuracy.
\end{itemize}

\textbf{Galileo Galilei — \textit{Geometric and Military Compass} (1597, publ. 1606)}

\begin{itemize}
    \item Multifunctional calculator using engraved arms — handled proportions, roots, and trajectory calculations.
    \item Employed for ballistics, architecture, and finance by soldiers, engineers, and merchants.
    \item Combined simplicity of form with conceptual sophistication — anticipating modern analog computing.
\end{itemize}

\textbf{Edmund Gunter — \textit{Canon Triangulorum} (1620)}

\begin{itemize}
    \item Published trigonometric tables with logarithmic values for sines and tangents.
    \item Invented the Gunter’s scale — a 2-foot ruler inscribed with logarithmic and trigonometric scales.
    \item Enabled calculation with dividers alone — a precursor to slide rules.
\end{itemize}

\textbf{William Oughtred — \textit{Clavis Mathematicae} (1631)}

\begin{itemize}
    \item Formalized symbolic algebra; adopted and adapted Viete’s notation.
    \item Invented the multiplication symbol ``$\times$'' and promoted compact mathematical syntax.
    \item Designed both circular and linear slide rules based on logarithmic principles — eliminating need for Gunter’s dividers.
\end{itemize}

\textbf{Johannes Kepler — \textit{Astronomia Nova} (1609) and \textit{Stereometria Doliorum} (1615)}

\begin{itemize}
    \item Defined elliptical planetary motion and introduced area laws using indivisibles.
    \item Developed volumetric integration techniques for solids of revolution (``citron,'' ``apple,'' ``wine barrel'').
    \item Unified geometric intuition with emerging integral methods — decades before formal calculus.
\end{itemize}




\newpage


\section*{IV. Historical Overview}

The early seventeenth century marked a transition from mathematical theory to \textit{mathematical instrumentation} — a shift from axioms to action, from rhetoric to rule.

\subsection*{1. Decimals Replace Duodecimals}

For centuries, sexagesimal and duodecimal fractions had dominated European calculation — a legacy of Babylonian astronomy and Roman commerce.

\begin{itemize}
    \item Stevin’s \textit{De Thiende} (1585) championed decimal fractions as universally applicable — for merchants, engineers, and clerks alike.
    \item His notation (circled numerals) made place value explicit, even if clunky by modern standards.
    \item The cultural effect was immense: decimals made arithmetic teachable — and mechanical.
\end{itemize}

Decimals became not just a numerical reform — but a cognitive one.

\subsection*{2. Logarithms Enter, Computation Collapses in Complexity}

Napier’s invention of logarithms collapsed multiplication into addition, squares into doubles.

\begin{itemize}
    \item His definition, based on dual motions (uniform and decaying), united geometry, kinematics, and ratios.
    \item Briggs’ collaboration yielded base-10 tables with intuitive endpoints: $\log 1 = 0$, $\log 10 = 1$.
    \item Bürgi independently developed logarithmic systems with high accuracy — confirming the idea’s inevitability.
\end{itemize}

Logarithms didn't just simplify math — they made precision portable.

\subsection*{3. From Tables to Tools: The Instrumental Turn}

With decimals and logs in hand, European practitioners built devices to externalize computation:

\begin{itemize}
    \item Galileo's geometric compass calculated roots, proportions, and interest by rotation.
    \item Gunter’s scale converted logarithms into a sliding physical measure — usable with dividers.
    \item Oughtred eliminated the dividers altogether, inventing the circular and linear slide rule.
\end{itemize}

Where before there were tables, now there were tools — and where there were tools, there were trades.

\subsection*{4. Algebra Becomes Symbolic — Slowly, Unevenly}

Though Viète had pioneered symbolic generality, full symbolic algebra remained rare:

\begin{itemize}
    \item Stevin used power indices, but not full abstraction.
    \item Oughtred introduced multiplication symbols and symbolic compactness, but notation remained non-standard.
    \item Negative and imaginary numbers remained marginalized — often seen as “fictitious” or “unreal.”
\end{itemize}

The symbolic revolution had begun — but was not yet hegemonic.

\subsection*{5. Indivisibles Emerge from Archimedean Shadows}

In pursuit of area, center, and volume — a new method was born from old constraints:

\begin{itemize}
    \item Stevin and Kepler treated areas as sums of infinitesimal slices — triangles, strips, and radii.
    \item Kepler defined planetary motion using “areas swept,” a proto-integral over time.
    \item Volumes were estimated by “slices” — turning ellipses into barrels, and circles into citrons.
\end{itemize}

This was not yet calculus — but it was no longer classical geometry.







\newpage



\section*{V. Problem–Solution Cycle}

\textbf{Problem 1: Napier’s Logarithmic Motion (c. 1614)}

\textit{Statement.} Define a logarithm geometrically using dual motions — one uniform, the other exponentially decaying.

\textit{Solution.}
\begin{itemize}
    \item Let point $P$ move along $AB$ starting at $A$, with velocity inversely proportional to its remaining distance to $B$.
    \item Let point $Q$ simultaneously move along $CE$ from $C$ with uniform velocity equal to $P$’s initial speed.
    \item Then, the distance $CQ$ is defined to be the logarithm of $PB$.
    \item In modern terms, this satisfies $\dfrac{dx}{dt} = -x$, $\dfrac{dy}{dt} = 1$ $\Rightarrow y = \ln x$ (scaled).
    \item Napier’s conception thus models logarithms through kinetic geometry — a bridge between arithmetic and motion.
\end{itemize}

\begin{center}
    \includegraphics[width=0.6\textwidth]{14pt1.png}

    \textit{FIG. 14.1: Napier’s definition of logarithm via linked motions}
\end{center}

\newpage

\textbf{Problem 2: Galileo’s Compass as a Mechanical Divider (c. 1597)}

\textit{Statement.} Use Galileo’s geometric compass to divide a line segment into five equal parts.

\textit{Solution.}
\begin{itemize}
    \item Open a pair of dividers to match the segment length $AB$.
    \item Set Galileo’s compass to span between two identical scale values divisible by 5 (e.g., 0–200 on each arm).
    \item Without altering the compass opening, move to the marks at 40 on both scales.
    \item The new distance between divider points gives $\tfrac{1}{5}$ of the original length.
    \item This mechanical proportionality replaces arithmetic division — a visual-logarithmic solution.
\end{itemize}

\begin{center}
    \includegraphics[width=0.5\textwidth]{14pt2.png}

    \textit{FIG. 14.2: Galileo’s geometric and military compass}
\end{center}

\newpage

\textbf{Problem 3: Stevin’s Center of Gravity via Parallelograms (c. 1586)}

\textit{Statement.} Prove that the center of gravity of a triangle lies on its median using only geometry.

\textit{Solution.}
\begin{itemize}
    \item Subdivide the triangle into a series of horizontal parallelograms of equal height, symmetric about the median.
    \item By symmetry, each parallelogram’s center lies on the median — and so does the composite figure’s center.
    \item Let the number of strips increase indefinitely; the area not captured by the parallelograms tends to zero.
    \item Thus, the triangle's true center of gravity lies on the same median line.
    \item This is an Archimedean limit argument — a precursor to integral statics.
\end{itemize}

\begin{center}
    \includegraphics[width=0.5\textwidth]{14pt3.png}

    \textit{FIG. 14.3: Stevin’s center of gravity via infinitesimal decomposition}
\end{center}

\newpage

\textbf{Problem 4: Area of a Circle via Indivisible Triangles (Kepler, c. 1609)}

\textit{Statement.} Approximate the area of a circle using infinitely many radial triangles.

\textit{Solution.}
\begin{itemize}
    \item Divide a circle into $n$ sectors, each approximated by a triangle of base $b_i$ (arc length) and height $r$ (radius).
    \item Sum of areas: $A \approx \tfrac{1}{2} \sum b_i r = \tfrac{1}{2} r \cdot C$ where $C$ is the circumference.
    \item Hence, $A = \tfrac{1}{2} r \cdot 2\pi r = \pi r^2$
    \item This infinitesimal triangulation anticipates integral calculus in area computation.
\end{itemize}

\begin{center}
    \includegraphics[width=0.4\textwidth]{14pt4.png}

    \textit{FIG. 14.4: Kepler’s approximation of circular area using radial slices}
\end{center}

\newpage

\textbf{Problem 5: Ellipse Area via Ordinate Compression (Kepler, c. 1615)}

\textit{Statement.} Derive the area of an ellipse from that of a circle via compression of ordinates.

\textit{Solution.}
\begin{itemize}
    \item Let a circle of radius $a$ be sliced vertically into infinitesimal strips.
    \item Compress each strip’s height by a factor of $\frac{b}{a}$ to form an ellipse.
    \item Each area becomes $dA_{\text{ellipse}} = \tfrac{b}{a} \cdot dA_{\text{circle}}$
    \item Therefore, total area: $A = \pi a^2 \cdot \frac{b}{a} = \pi ab$
    \item This reasoning generalizes area by transformation — a hallmark of modern integral calculus.
\end{itemize}

\begin{center}
    \includegraphics[width=0.4\textwidth]{14pt5.png}

    \textit{FIG. 14.5: Kepler’s derivation of ellipse area by stripwise compression}
\end{center}



\newpage

\section*{VI. Decline and Disruption: Limits of the Instrumental Age}

The era of early modern problem solvers brought new tools, tables, and trigonometric techniques — but it could not yet unify them into a general theory.

\subsection*{I. Computation Without Concept}

From Stevin’s decimals to Napier’s logarithms, the period was computationally powerful — but conceptually fragmented.

\begin{itemize}
    \item Logarithms simplified multiplication but lacked a unified exponential model.
    \item Decimals became widespread — but were viewed as notation, not number theory.
    \item Geometric compasses could perform operations — but did not yield general laws.
\end{itemize}

It was a golden age of answers — but not yet of unification.

\subsection*{II. Symbols Without Standardization}

While Viète and Oughtred advanced symbolic algebra, the field remained deeply inconsistent:

\begin{itemize}
    \item Multiple notations for powers, roots, equality, and negatives coexisted across countries.
    \item Even the multiplication sign ``$\times$'' was only recently introduced and not universally accepted.
    \item Algebra remained largely rhetorical outside specialist circles.
\end{itemize}

Notation accelerated thought — but still had no grammar.

\subsection*{III. Indivisibles Without Foundations}

Kepler and Stevin used indivisibles to approximate areas and volumes — but lacked a rigorous notion of the infinite:

\begin{itemize}
    \item The idea of a “strip” or “slice” worked in practice, but not in principle.
    \item Archimedean exhaustion was invoked, but not proven.
    \item There was no limit theory — no $\varepsilon$-$\delta$, no convergence, no real number system.
\end{itemize}

They had glimpsed integration — but not continuity.

\subsection*{IV. Geometry Separated from Algebra}

Though geometry thrived in tools (compasses, sectors) and projections (Keplerian ellipses, Galileo’s parabolas):

\begin{itemize}
    \item It was not yet analytic — Cartesian coordinates did not exist.
    \item Algebra could not describe curves, nor geometry express functions.
    \item The two great languages of mathematics remained mutually unintelligible.
\end{itemize}

It was not ignorance — it was isolation.

\subsection*{V. Astronomy and Physics Strained the Framework}

Kepler’s ellipses, Galileo’s accelerations — all demanded a new mathematics.

\begin{itemize}
    \item The ellipse resisted circular explanation.
    \item Constant acceleration defied finite arithmetic.
    \item Planetary laws were geometric in shape but algebraic in dependence.
\end{itemize}

The instruments had done their job — they had made the classical system untenable.

\subsection*{Conclusion: Awaiting Unification}

By 1640:

\begin{itemize}
    \item Computation had advanced far ahead of theory.
    \item Instruments could perform what minds could not yet prove.
    \item Ideas like limit, derivative, function, and integral hovered in the air — but had no name, no symbol, no proof.
\end{itemize}

The century had solved many problems —  
But it had not yet invented the mathematics that would explain why those solutions worked.





\newpage

\section*{VII. Closing Dialectic}

\textbf{Summary}\\

In the age of Stevin, Napier, and Kepler, mathematics became a tool of action.\\

\begin{itemize}
    \item Decimals were not metaphysical — they were merchantable.
    \item Logarithms were not proven — they were published.
    \item Compasses were not symbols — they were steel.
    \item Triangles were not axioms — they were artillery.
    \item Indivisibles were not foundations — they were slices.
\end{itemize}

This was not the age of rigor.  
It was the age of reckoning.\\

They calculated faster than they reasoned.  
They drew further than they proved.  
They measured beyond the reach of their definitions.\\

And yet —\\

They multiplied without tables.  
They approximated without calculus.  
They proved with balances, rods, and wedges.\\

\textbf{They knew what worked, even if they could not yet say why.}\\

\textbf{Comparative Mathematical Cosmologies}\\

\begin{itemize}
    \item \textbf{Greek:} Mathematics as essence. Geometry as deduction. Infinity as paradox.
    \item \textbf{Islamic:} Mathematics as inheritance. Algebra as order. Infinity as discrete.
    \item \textbf{Renaissance:} Mathematics as recovery. Number as symbol. Infinity as suggestion.
    \item \textbf{Early Modern:} Mathematics as instrument. Calculation as craft. Infinity as slice.
\end{itemize}

Each age did not merely compute —  
It recalibrated the meaning of computation.\\

\textbf{Exit Prompt}\\

You are Napier. Or Stevin. Or a naval cartographer holding a scale.\\

You do not have a derivative.  
You do not know the integral.  
You have no formal definition of function, nor of convergence, nor of limit.\\

But you hold the compass.  
You slice the circle.  
You compress the ellipse.\\

And it works.\\

You do not yet see the future —  
But you lay its groundwork, wedge by wedge, digit by digit, rod by rod.\\

\textbf{What instrument do you build?}  
\textbf{What approximation do you dare to trust?}  
\textbf{And who, one generation later, will see the theorem inside your tool?}



\end{document}