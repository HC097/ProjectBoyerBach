\documentclass[9pt]{article}
\usepackage{amsmath, amssymb, geometry, graphicx}
\usepackage{titlesec}
\titleformat{\section}[block]{\large\bfseries}{\thesection}{1em}{}
\titleformat{\subsection}[runin]{\bfseries}{}{0pt}{}[.]

\begin{document}

\begin{center}
\Large\textbf{Chapter 21 – Algebra} \\
\large Harley Caham Combest \\
\large Su2025 MATH4991 Lecture Notes – Mk1
\end{center}

\vspace{1em}

\section*{I. Cultural Invocation}

\begin{itemize}
  \item \textbf{Civilization:} Nineteenth-century Europe — Britain, France, Germany — contending with the legacy of Enlightenment precision and Romantic abstraction.
  \item \textbf{Time Span:} Late 18th century to early 20th century
  \item \textbf{Epochal Axes:} Symbolism, Structure, Operation, Logic, Generalization
  \item \textbf{Figures:} Peacock, De Morgan, Boole, Hamilton, Grassmann, Cayley, Sylvester, Dedekind, Kronecker, Frege, Peano
\end{itemize}

Algebra is the language mathematics speaks to itself.\\

\noindent
It began as calculation — rules for solving equations.  
But in the 19th century, it became something else:  
Not just a tool, but a mirror — dominated by British (and non-trivially Irish) linguists reflecting the deeper logic beneath all operations.\\

\noindent
What is a number?  \\
What is a valid operation?  \\
Can structure exist without content?\\

These were no longer technical questions — they were foundational.\\

\begin{itemize}
  \item George Peacock asked whether algebra is formal or arithmetical.
  \item Boole used algebra to encode logic itself.
  \item Hamilton discovered a new kind of number — where $ij = k$, but $ji = -k$.
  \item Grassmann defined vectors not from geometry, but from algebraic combination.
  \item Cayley turned matrices from bookkeeping devices into abstract operators.
  \item Dedekind asked: what counts as a number system at all?
  \item Peano formalized the natural numbers into logical axioms — arithmetic made symbolic law.
\end{itemize}

Algebra became structural.\\

\noindent
Where arithmetic counted, algebra now classified.  
Where symbols once represented, they now generated.\\

This was not a refinement. It was a reformation.\\

\newpage

\section*{II. Big Pictures}

Algebra in the 19th century did not merely expand.  
It changed direction — and then changed form.\\

\noindent
This was not just the solving of harder equations.  
It was the invention of entire new kinds of number, operation, and system.\\

\noindent
Here are the major currents that reshaped the field:

\bigskip

\textbf{1. Symbolic Formalism (Peacock, De Morgan)}

Algebra as manipulation of symbols.  
Rules are valid not because they reflect numbers, but because they are internally consistent.  
Algebra begins to detach from arithmetic.

\bigskip

\textbf{2. Logic as Algebra (Boole, Peirce, Frege)}

Algebra becomes a language for logic itself.  
Statements become equations. Truth becomes symbolically computed.  
This is not just number applied to thought — it is thought revealed as structure.

\bigskip

\textbf{3. New Number Systems (Hamilton, Grassmann)}

Algebra is no longer bound to the real or complex numbers.  
Hamilton invents quaternions — non-commutative number systems.  
Grassmann defines vector spaces without geometry.  
Numbers multiply — and with them, algebra splits open.

\bigskip

\textbf{4. Algebraic Structures (Cayley, Sylvester)}

Operations are abstracted.  
Matrices, groups, and linear operators gain lives of their own.  
Algebra shifts from solving equations to defining transformation.

\bigskip

\textbf{5. Foundations and Axioms (Dedekind, Peano, Frege)}

What is a number?  
Peano formalizes natural numbers from axioms.  
Dedekind defines number systems through sets and mappings.  
Algebra approaches logic — and asks what makes a system coherent.

\bigskip

\textbf{6. The Algebra-Geometry Bridge}

Algebra becomes a language for describing space.  
The work of Clebsch, Cayley, and others turns geometric intuition into symbolic computation.  
This sets the foundation for the rise of algebraic geometry in the next century.

\bigskip

Each of these currents moved in parallel — but together they redefined the field.  
Algebra was no longer just about solving.  
It was about building: systems, structures, and the languages to describe them.

\newpage

\section*{III. Epochal Outline — Algebra Through Iteration}

Algebra is not a single method.  
It is a series of redefinitions — each one shifting the boundary of what counts as number, operation, and form.

Here are seven major phases in algebra’s transformation during the 19th century:

\bigskip

\textbf{1. Arithmetical Symbolism} \hfill \textit{(Peacock, De Morgan)}

\begin{itemize}
  \item Algebra begins as generalized arithmetic.
  \item Symbols stand for quantities.  
  \item Equations are solved, but not yet abstracted.
  \item Laws mirror numerical behavior: commutativity, associativity, distributivity.
\end{itemize}

\bigskip

\textbf{2. Formal Algebra} \hfill \textit{(Peacock, Boole)}

\begin{itemize}
  \item Symbols need not refer to numbers.
  \item Operations are defined formally — independent of interpretation.
  \item Algebra becomes self-referential: a system governed by internal law.
\end{itemize}

\bigskip

\textbf{3. Logic and Algebra Merge} \hfill \textit{(Boole, Peirce, Frege)}

\begin{itemize}
  \item Algebra expresses logic.
  \item Statements become equations; operations become inference.
  \item Frege systematizes propositional logic using algebraic form.
\end{itemize}

\bigskip

\textbf{4. Extension of Number} \hfill \textit{(Hamilton, Grassmann)}

\begin{itemize}
  \item New number systems arise: complex, quaternions, vectors.
  \item Algebra becomes a way to invent structure.
  \item Non-commutativity and non-associativity are explored, not rejected.
\end{itemize}

\bigskip

\textbf{5. Abstract Structure} \hfill \textit{(Cayley, Sylvester)}

\begin{itemize}
  \item Focus shifts from solving to defining systems.
  \item Matrices, linear transformations, and groups are treated on their own terms.
  \item Algebra is now the study of operations and their symmetries.
\end{itemize}

\bigskip

\textbf{6. Arithmetic Foundations} \hfill \textit{(Dedekind, Peano, Frege)}

\begin{itemize}
  \item Number itself is redefined.
  \item Peano axioms define $\mathbb{N}$ from logic.
  \item Dedekind constructs $\mathbb{Q}$ and $\mathbb{R}$ from set-theoretic mappings.
  \item Frege attempts a full logicist foundation.
\end{itemize}

\bigskip

\textbf{7. Algebra Meets Geometry} \hfill \textit{(Clebsch, Cayley)}

\begin{itemize}
  \item Algebra becomes the language of curves and surfaces.
  \item Equations define geometric objects; transformations reveal symmetry.
  \item The boundary between algebra and geometry begins to dissolve.
\end{itemize}

\newpage

\section*{IV. Iterative Approaches to a Core Problem}

A guiding question in algebra:

\begin{quote}
\textbf{Given an equation, what does it mean to solve it?}
\end{quote}

\noindent
This is not a constant question.  
Each era of algebra would define “solving” differently — and reshape what the equation even is.

\bigskip

\textbf{1. Arithmetical Algebra} \hfill \textit{(Peacock)}

\begin{itemize}
  \item An equation is a numerical statement.
  \item Solving means finding a number that satisfies it.
  \item The symbols behave like numbers — and must obey their rules.
\end{itemize}

\bigskip

\textbf{2. Formal Symbolic Algebra} \hfill \textit{(De Morgan, Boole)}

\begin{itemize}
  \item An equation is a formal expression.
  \item Solving means transforming symbols according to internal rules — not interpretation.
  \item The solution is structural, not numerical.
\end{itemize}

\bigskip

\textbf{3. Algebra as Logic} \hfill \textit{(Boole, Frege)}

\begin{itemize}
  \item An equation encodes a logical relation.
  \item Solving means checking truth-preserving inference.
  \item Variables stand for propositions; operations stand for reasoning.
\end{itemize}

\bigskip

\textbf{4. Algebra as Structure} \hfill \textit{(Hamilton, Cayley)}

\begin{itemize}
  \item An equation may live in a non-numerical system — like matrices or quaternions.
  \item Solving means finding elements within a defined structure that satisfy a relation.
  \item The solution is contextual: it depends on the system’s laws.
\end{itemize}

\bigskip

\textbf{5. Algebra as Foundation} \hfill \textit{(Dedekind, Peano)}

\begin{itemize}
  \item The equation is part of an axiomatic system.
  \item Solving is derivability: what follows from definitions and axioms.
  \item Algebra becomes a logic of system-building.
\end{itemize}

\bigskip

\textit{From number to symbol.  
From computation to transformation.  
From solving to structuring.}\\

\noindent
Algebra no longer solves equations.  
It defines the worlds in which equations exist.

\newpage

\section*{V. Closing Dialectic}

Algebra did not evolve by extension.  
It evolved by inversion.\\

What began as the art of solving equations  
became the act of defining what an equation is.\\

\bigskip

The 19th century transformed algebra from a practice to a philosophy:
\begin{itemize}
  \item Boole made it logical.
  \item Hamilton made it spatial.
  \item Cayley made it operational.
  \item Dedekind and Peano made it foundational.
\end{itemize}

What unified these movements was not method — but intent.\\

Algebra ceased to be about answers.  
It became a theory of what kinds of questions were possible.\\

\bigskip

The core reversal:\\
\begin{quote}
Algebra no longer described numbers.  
It described structures that included numbers as a special case.
\end{quote}

This was not a collapse.  
It was a generalization — radical and abstract, but coherent.\\

What followed would not be clarity, but power.  
The 20th century would inherit this abstraction — and extend it toward rings, fields, categories, modules.\\

But that story begins only after algebra stopped being about quantity —  
and began being about form.\\


\end{document}