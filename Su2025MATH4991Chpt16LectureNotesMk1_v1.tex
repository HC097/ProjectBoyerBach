\documentclass[9pt]{article}
\usepackage{amsmath, amssymb, geometry, graphicx}
\usepackage{titlesec}
\titleformat{\section}[block]{\large\bfseries}{\thesection}{1em}{}
\titleformat{\subsection}[runin]{\bfseries}{}{0pt}{}[.]

\begin{document}

\begin{center}
\Large\textbf{Chapter 16 – British Techniques and  Continental Methods} \\
\large Harley Caham Combest \\
\large Su2025 MATH4991 Lecture Notes – Mk1
\end{center}

\vspace{1em}

\section*{I. Cultural Invocation}

\begin{itemize}
  \item \textbf{Civilization:} Late Seventeenth to Early Eighteenth Century Europe
  \item \textbf{Time Period:} c.~1650--1725
  \item \textbf{Epochs:} English Restoration, Dutch Decline, Newtonian Synthesis
  \item \textbf{Figures:} John Wallis, James Gregory, Isaac Barrow, Isaac Newton, Abraham De Moivre, Gottfried Wilhelm Leibniz
\end{itemize}

The infinite ceased to be taboo.
\vspace{1pt}
The legacy of Descartes passed into the hands of Wallis, Barrow, and Newton—where geometry met flux, and proof met power.
\vspace{1pt}
As the Continent soared with curves, Britain grounded itself with rigor: proportions became series, indivisibles became fluxions, and the spiral of history bent toward calculus.
\vspace{1pt}
\emph{Who may claim invention? What is the rigor of infinitesimals? Can geometry survive under infinite descent?}
\vspace{1pt}
This was the age where mathematics became empire.

\begin{center}
    \includegraphics[scale=.1]{cover1.png}

    \textit{Geopolitical Map of World, 1700AD}
\end{center}


\newpage

\section*{II. Faces of the Era}

\subsection*{John Wallis (1616–1703)}
\begin{itemize}
  \item Introduced the symbol $\infty$; expanded Cavalieri’s indivisibles arithmetically.
  \item Developed Wallis’s product for $\pi$ and early integral formulas.
  \item Advanced analytic geometry and infinite series in England.
\end{itemize}

\subsection*{Isaac Barrow (1630–1677)}
\begin{itemize}
  \item First to articulate the inverse relation between integration and differentiation.
  \item Taught Newton and edited early lectures on geometric calculus.
  \item Reconciled geometry with early infinitesimal reasoning.
\end{itemize}

\subsection*{James Gregory (1638–1675)}
\begin{itemize}
  \item Anticipated Taylor series and found exact expressions for arctangent integrals.
  \item Studied quadratures of curves; contributed to early convergence theory.
  \item Introduced series expansions for inverse trigonometric functions.
\end{itemize}

\subsection*{Isaac Newton (1642–1727)}
\begin{itemize}
  \item Invented fluxional calculus and infinite series for fractional exponents.
  \item Formulated the laws of motion and universal gravitation in the \textit{Principia}.
  \item Developed the binomial theorem and the method of fluxions.
\end{itemize}

\subsection*{Gottfried Wilhelm Leibniz (1646–1716)}
\begin{itemize}
  \item Independently developed differential and integral calculus.
  \item Created the notation $dx$, $\int$, and symbolic form of the derivative.
  \item Founded the analytic style that dominated Continental mathematics.
\end{itemize}

\subsection*{Jacques (James) Bernoulli (1654–1705)}
\begin{itemize}
  \item Founded the theory of probability in \textit{Ars Conjectandi}.
  \item Introduced Bernoulli numbers and the Law of Large Numbers.
  \item Investigated the logarithmic spiral and early combinatorics.
\end{itemize}

\subsection*{Johann (Jean) Bernoulli (1667–1748)}
\begin{itemize}
  \item Prolific expositor and extender of Leibnizian calculus.
  \item Advanced differential equations, brachistochrones, and exponential calculus.
  \item Mentored Euler and disseminated analysis across Europe.
\end{itemize}

\subsection*{Abraham De Moivre (1667–1754)}
\begin{itemize}
  \item Developed analytic trigonometry using complex numbers.
  \item Authored \textit{The Doctrine of Chances}, foundational to probability theory.
  \item Anticipated the normal distribution and Stirling’s approximation.
\end{itemize}

\subsection*{Brook Taylor (1685–1731)}
\begin{itemize}
  \item Formulated the Taylor series for function expansion.
  \item Studied curvature, oscillations, and perspective geometry.
  \item Bridged Newtonian fluxions and modern analytic mechanics.
\end{itemize}

\subsection*{Alexis Claude Clairaut (1713–1765)}
\begin{itemize}
  \item Founded solid analytic geometry and studied space curves.
  \item Extended curvature and potential theory to three dimensions.
  \item Applied calculus to gravitational theory and the figure of the Earth.
\end{itemize}


\newpage

\section*{III. Works of the Era}

\subsection*{John Wallis — \textit{Arithmetica Infinitorum} (1655)}
\begin{itemize}
  \item Replaced Cavalieri’s indivisibles with arithmetical series.
  \item Extended integration to general powers of $x$ using interpolation.
  \item Introduced infinite products and early conceptions of area under curves.
\end{itemize}

\subsection*{John Wallis — \textit{Treatise of Algebra, Both Historical and Practical} (1685)}
\begin{itemize}
  \item Merged symbolic manipulation with historical development.
  \item Asserted English priority in algebraic methods over Descartes.
  \item Provided foundational reference for Newton and other English mathematicians.
\end{itemize}

\subsection*{Isaac Newton — \textit{De analysi per aequationes numero terminorum infinitas} (1669 / 1711)}
\begin{itemize}
  \item Introduced Newton’s general method of infinite series expansion.
  \item Showed that curves could be expressed and integrated via power series.
  \item Marked the beginning of Newton’s analytic calculus.
\end{itemize}

\subsection*{Isaac Newton — \textit{Philosophiae Naturalis Principia Mathematica} (1687)}
\begin{itemize}
  \item Unified physics through geometric derivation of universal gravitation.
  \item Formalized laws of motion and mathematical astronomy.
  \item Contained hidden uses of calculus under geometric language.
\end{itemize}

\subsection*{G. F. A. de L’Hospital — \textit{Analyse des infiniment petits} (1696)}
\begin{itemize}
  \item First textbook on differential calculus using Leibniz’s notation.
  \item Contained l’Hospital’s Rule, originating from Johann Bernoulli.
  \item Made calculus accessible to the broader European mathematical community.
\end{itemize}

\subsection*{Isaac Newton — \textit{De Quadratura Curvarum} (1704)}
\begin{itemize}
  \item Developed integration techniques explicitly tied to fluxions.
  \item Introduced the notion of the “ultimate ratio” — a precursor to limits.
  \item Signaled Newton’s response to the rising popularity of Leibniz’s methods.
\end{itemize}

\subsection*{Jacques Bernoulli — \textit{Ars Conjectandi} (1713)}
\begin{itemize}
  \item Founded the modern theory of probability.
  \item Introduced Bernoulli numbers and the Law of Large Numbers.
  \item Unified combinatorics, expectation, and mathematical rigor.
\end{itemize}

\subsection*{Isaac Newton — \textit{Methodus Fluxionum et Serierum Infinitorum} (1736)}
\begin{itemize}
  \item Posthumous English publication of Newton’s method of fluxions.
  \item Developed rules for differentiation and their inverse (integration).
  \item Presented Newtonian calculus more directly than the \textit{Principia}.
\end{itemize}

\subsection*{Abraham De Moivre — \textit{Miscellanea Analytica} (1730)}
\begin{itemize}
  \item Applied complex numbers to trigonometric expansions.
  \item Advanced probability theory and approximations (precursor to the normal distribution).
  \item Introduced what would become De Moivre’s Theorem.
\end{itemize}

\subsection*{Colin Maclaurin — \textit{Treatise of Fluxions} (1742)}
\begin{itemize}
  \item Most rigorous defense of Newtonian fluxions against Berkeley’s critique.
  \item Grounded calculus in classical geometric logic.
  \item Preserved Newton’s framework during the rise of Continental analysis.
\end{itemize}

\vspace{1em}
\noindent
\textbf{Note: Leibniz's Foundational Papers on Calculus}

\noindent
Although not included among the full treatises listed above, the following works by Gottfried Wilhelm Leibniz merit special recognition as the earliest published expositions of differential and integral calculus:

\begin{itemize}
  \item \textit{Nova Methodus pro Maximis et Minimis, itemque Tangentibus...} (\textit{Acta Eruditorum}, 1684)
  \item \textit{De Geometria Recondita et Analysi Indivisibilium atque Infinitorum} (\textit{Acta Eruditorum}, 1686)
\end{itemize}

These journal papers (in The Acts of the Erudite [1st Scientific Journal of German-Speaking Europe]) introduced the $d$ notation, the integral symbol $\int$, and the philosophical framing of calculus as a symbolic, general method. Though brief, they seeded a revolution in mathematical practice and notation across the Continent.

\newpage

\section*{IV. Historical Overview}

The birth of calculus was not a singular moment but a divided unfolding—one British, geometric, and guarded; the other Continental, symbolic, and bold.

\subsection*{1. England’s Geometric Conservatism}
\begin{itemize}
  \item Newton's calculus remained veiled in geometry, driven by fluxions, ratios, and limits concealed under classical form.
  \item Barrow's rigor and Wallis's infinite products shaped a method of reasoning that valued continuity with Euclid over symbolic convenience.
  \item British thinkers distrusted algebraic abstraction—preferring the solidity of geometric construction.
\end{itemize}

\subsection*{2. The Rise of Symbolic Calculus in Europe}
\begin{itemize}
  \item Leibniz introduced $dx$, $dy$, and $\int$ as autonomous symbols of change and summation.
  \item Johann Bernoulli translated these into usable algorithms—turning calculus into a system.
  \item L’Hospital's textbook formalized these tools and ensured their spread from Paris to Basel.
\end{itemize}

\subsection*{3. Probability and Analysis Merge}
\begin{itemize}
  \item Jacques Bernoulli’s \textit{Ars Conjectandi} unified combinatorics, expectation, and inductive reasoning.
  \item De Moivre extended this into continuous forms—paving the path to Gaussian distributions and statistical calculus.
\end{itemize}

\subsection*{4. Controversy over Foundations}
\begin{itemize}
  \item Berkeley’s \textit{The Analyst} attacked the logical coherence of fluxions as “ghosts of departed quantities.”
  \item Maclaurin responded with classical rigor—attempting to ground Newton’s method in geometric first principles.
  \item Meanwhile, Continental analysts moved forward pragmatically, refining convergence and approximation through series and limits.
\end{itemize}

\subsection*{5. The Divide Solidifies}
\begin{itemize}
  \item By 1750, the Continental school—Leibniz, the Bernoullis, Euler—had overtaken British methods in clarity and computational power.
  \item British loyalty to Newton’s fluxional method, though noble, delayed full engagement with modern analysis for nearly a century.
\end{itemize}

\subsection*{6. Geometry, Revived but Redefined}
\begin{itemize}
  \item Analytic geometry expanded into three dimensions with Clairaut and Hermann.
  \item Newton’s solid curves were reinterpreted as algebraic loci; surfaces began to be classified symbolically.
  \item Geometry had not died—it had transformed into algebraic structure.
\end{itemize}

\vspace{1em}
\noindent
\textbf{Note: Newtonian vs. Leibnizian Calculus — A Comparative Overview}

\begin{center}
\begin{tabular}{|p{6.5cm}|p{6.5cm}|}
\hline
\textbf{Newtonian Calculus} & \textbf{Leibnizian Calculus} \\
\hline
\textit{Fluxions and fluents}: quantities changing over time & \textit{Differentials and integrals}: infinitesimal increments and summation \\
\hline
Grounded in geometry and kinematics & Grounded in symbolic analysis \\
\hline
Notation: $\dot{x}$ (fluxion), $\ddot{x}$ (second fluxion) & Notation: $dx$, $dy$, $\int$ (integral) \\
\hline
Defined change as a ratio of vanishing quantities ("ultimate ratios") & Treated $dx$ and $dy$ as actual algebraic entities manipulable in equations \\
\hline
First developed in \textit{De Analysi} and \textit{Principia}; method often concealed in geometric form & First published in 1684–1686 in \textit{Acta Eruditorum}; method explicitly symbol-based \\
\hline
Developed independently in England; spread slowly due to lack of clear exposition & Disseminated quickly through Europe via L’Hospital, Bernoulli, and Euler \\
\hline
Cautious about symbolic generalization; prioritized rigor via classical geometry & Embraced generalization; prioritized versatility and computation \\
\hline
\end{tabular}
\end{center}

\vspace{0.5em}
\noindent
Despite these differences, both systems captured the same underlying truths. They approached limits, tangents, and areas from different angles—but converged in power. The calculus was not born twice. It was revealed in two dialects of genius.


\newpage

\section*{V. Problem--Solution Cycle}

\subsection*{Problem 1: Subtangent by Infinitesimal Triangle (Barrow, 1670s)}
\textbf{Statement.} Given Barrow’s triangle \( \triangle MRN \) along a curve with known ordinate \( m \), derive the subtangent \( t \) using triangle similarity. \\
\textbf{Solution.}
\begin{itemize}
  \item Let \( a \) and \( e \) be the vertical and horizontal legs of triangle \( MRN \), with \( m \) the ordinate at \( M \).
  \item By similarity: \( \frac{a}{e} = \frac{m}{t} \Rightarrow t = \frac{e \cdot m}{a} \).
  \item This is an early geometric precursor to the concept of the derivative.
\end{itemize}
\begin{center}
\includegraphics[width=0.4\textwidth]{16pt1.png} \\
\textbf{FIG. 16.1:} Barrow’s subtangent triangle via infinitesimals.
\end{center}

\newpage

\subsection*{Problem 2: Deriving Gregory’s Series for \( \arctan x \) (Gregory, 1671)}
\textbf{Statement.} Expand \( \frac{1}{1 + x^2} \) as a power series and integrate to obtain Gregory’s series. \\
\textbf{Solution.}
\begin{itemize}
  \item Divide: \( \frac{1}{1 + x^2} = 1 - x^2 + x^4 - x^6 + \cdots \)
  \item Integrate term-by-term:
  \[
  \arctan x = x - \frac{x^3}{3} + \frac{x^5}{5} - \frac{x^7}{7} + \cdots
  \]
  \item Converges for \( |x| \leq 1 \); anticipates later Taylor expansions.
\end{itemize}

\newpage

\subsection*{Problem 3: Newton’s Ultimate Ratio of Powers (Newton, 1671)}
\textbf{Statement.} Use binomial expansion on \( (x + o)^n - x^n \) and find the ultimate ratio as \( o \to 0 \). \\
\textbf{Solution.}
\begin{itemize}
  \item Expand: \( (x + o)^n = x^n + n x^{n-1} o + \cdots \)
  \item Subtract \( x^n \), divide by \( o \), let \( o \to 0 \):
  \[
  \lim_{o \to 0} \frac{(x + o)^n - x^n}{o} = n x^{n-1}
  \]
  \item A geometric justification of the derivative.
\end{itemize}

\newpage

\subsection*{Problem 4: Newton’s Polygon and the Folium (Newton, 1676)}
\textbf{Statement.} Analyze the folium \( x^3 + y^3 = 3axy \) using Newton’s lattice diagram. \\
\textbf{Solution.}
\begin{itemize}
  \item Organize terms by total degree into diagonal lattice segments.
  \item Each segment provides a local approximation (branch) of the curve.
  \item This anticipates Newton’s polygon method for singularities.
\end{itemize}
\begin{center}
\includegraphics[width=0.5\textwidth]{16pt2.png} \\
\textbf{FIG. 16.2:} Newton’s diagram for the folium of Descartes.
\end{center}

\newpage

\subsection*{Problem 5: Triangle-Arc Ratio and \( \frac{d\theta}{dx} \) (Pascal–Leibniz, 1680s)}
\textbf{Statement.} Use a circle diagram to show that \( \frac{AD}{DI} = \frac{EE}{RR} \approx \frac{d\theta}{dx} \). \\
\textbf{Solution.}
\begin{itemize}
  \item Treat \( EE \) as arc, \( RR \) as projection; triangle \( ADI \) is infinitesimal.
  \item Ratio approximates:
  \[
  \frac{1}{\sin \theta} = \frac{d\theta}{dx}
  \]
  \item Visual bridge from geometry to differential identities.
\end{itemize}
\begin{center}
\includegraphics[width=0.4\textwidth]{16pt3.png} \\
\textbf{FIG. 16.3:} Pascal’s triangle of differential ratios.
\end{center}

\newpage

\subsection*{Problem 6: Leibniz’s Product Rule from Differentials (Leibniz, 1684)}
\textbf{Statement.} Derive \( d(xy) = x \, dy + y \, dx \) by expanding \( (x + dx)(y + dy) \). \\
\textbf{Solution.}
\begin{itemize}
  \item Expand: \( (x + dx)(y + dy) - xy = x\,dy + y\,dx + dx\,dy \)
  \item Discard \( dx\,dy \) as second-order.
  \item Final rule: \( d(xy) = x\,dy + y\,dx \)
\end{itemize}

\newpage

\subsection*{Problem 7: Integral of \( x^x \) via Series (Bernoulli, 1690s)}
\textbf{Statement.} Express \( x^x \) as a power series and integrate \( \int_0^1 x^x dx \). \\
\textbf{Solution.}
\begin{itemize}
  \item Use: \( x^x = e^{x \ln x} = \sum \frac{(x \ln x)^n}{n!} \)
  \item Integrate term-by-term:
  \[
  \int_0^1 x^x dx = 1 - \frac{1}{2^2} + \frac{1}{3^3} - \cdots
  \]
  \item Early use of exponentials and infinite series.
\end{itemize}

\newpage

\subsection*{Problem 8: Permutations from Conditional Probability (De Moivre, 1711)}
\textbf{Statement.} Derive the number of 2-letter permutations using probability logic. \\
\textbf{Solution.}
\begin{itemize}
  \item \( P(\text{first letter}) = 1/6 \), \( P(\text{second}) = 1/5 \)
  \item Product = \( 1/30 \Rightarrow 30 \) permutations
  \item General form:
  \[
  P(n, r) = \frac{n!}{(n - r)!}
  \]
\end{itemize}

\newpage

\section*{VII. Closing Dialectic}

The calculus was not invented. It was uncovered—twice.

\vspace{1cm}
\noindent
Newton, in solitude, conjured fluxions from motion and geometry. Leibniz, in correspondence, spun differentials from language and logic. They did not merely compute—they redefined what it meant to know change.
\vspace{1cm}

\noindent
What united them was not method, but destiny:  
A universe in motion required a mathematics that could speak its grammar.

\begin{itemize}
  \item Newton fused Euclid and motion, offering a calculus of force.
  \item Leibniz fused logic and infinity, offering a calculus of thought.
  \item The Bernoullis transformed that calculus into technique.
  \item De Moivre and Bernoulli seeded probability and the shape of chance.
  \item Maclaurin and Clairaut defended structure at the edge of rigor.
\end{itemize}

\noindent
And yet, neither fluxions nor differentials escaped unscathed.  
The infinite still lacked foundation.  
The derivative still hid its definition.  
The integral still demanded formal precision.

\noindent
The dialectic remains:

\begin{itemize}
  \item Can motion be mathematized without first understanding limit?
  \item Can symbolism yield truth when its foundations remain implicit?
  \item Can two systems born in rivalry give birth to one enduring legacy?
\end{itemize}

\noindent
Calculus did not begin with certainty. It began with power.  
And it would take a century of correction, translation, and confrontation  
before its fire would be tempered into clarity.

\begin{center}
    \textit{From fluxions to functions. From rivalry to reason.} \\
    \textit{This was the birth of the modern.}
\end{center}


\end{document}
